\documentclass{article}

\usepackage[T2A]{fontenc}
\usepackage[utf8]{inputenc}
\usepackage[russian]{babel}

\linespread{1.1}
\setlength{\parskip}{1em}
\usepackage[left=1cm,right=1cm,top=1cm,bottom=2cm]{geometry}

\begin{document}

\footnotemark[1]{ "Объектом знания должно быть нечто устойчивое и общее. А так как чувственно воспринимаемые предметы преходящи и единичны, то предметом знания могут быть только эйдосы".}

\footnotemark[2]{ "Раз существует эйдос для предмета, то должен существовать эйдос и для отрицания предмета". Например, если есть эйдос для человека, то должен быть эйдос и для нечеловека.}

\footnotemark[3]{ "Эйдосы должны существовать для предметов, которые могут всецело исчезнуть, но сохраниться в человеческой мысли". Но то же самое можно отнести и к единичным предметам и утверждать, что по той же причине для каждого из них должен существовать особый эйдос.}

\footnotemark[4]{ "Есть эйдос реального человека. Есть эйдос абстрактного человека. Так как между реальным и абстрактным человеком есть нечто общее, то у этого общего тоже есть эйдос". Тут начинается цепная реакция: сравнение порождённого эйдоса и изначальным порождает новый эйдос - и так до бесконечности.}

\footnotemark[5]{ Я не понимаю, как оно так противоречит, ну да ладно.}

\footnotemark[6]{ Я пропустил абзац: совсем непонятно.}

\footnotemark[7]{ Фигня какая-то. Наверняка Аристотель под "математическими объектами" имел в виду что-то своё, древнее. Со стороны современной математики, и линия, и число - это математический объект.}

\end{document}

