\documentclass{article}

\usepackage[T2A]{fontenc}
\usepackage[utf8]{inputenc}
\usepackage[russian]{babel}

\linespread{1.1}
\setlength{\parskip}{1em}
\usepackage[left=1cm,right=1cm,top=1cm,bottom=2cm]{geometry}

\begin{document}

\footnotemark[1]{ Я буду стараться использовать термин "эйдос". Не давая прямого определения, я перечислю ряд близких к нему слов: "вид", "эталон", "образец", "чертёж". Эйдос нематериален. Эйдос отличается от идеи тем, что идея обобщает несколько вещей, а эйдос выделяет конкретику. Идея - это определение сути, эйдос - определение признаков. Скульптор может воспринять эйдос мысленно и двигать к нему свою скульптуру.}

\footnotemark[2]{ Иногда почему-то употребляют схожий термин "forms", но это не одно и то же. }

\footnotemark[3]{ Например, эйдос квадрата уникален. А вот математических квадратов существует много. }

\footnotemark[4]{ Ранее по тексту термин "единое" не разъясняется. О нём будет много сказано в пятой книге. Пока можно считать "единое" самой высшей сущностью.}

\footnotemark[5]{ В "Метафизике" я не увидел явного определения причастности, но по тексту встречаются определяющие примеры. Очень грубо и кратко: причастный - это что-то вроде подкласса. Хорошее слово - "сиюминутный". <ul><li><ul><li>"Человек" причастен "живому существу".</li><li>"Человек" - подкласс "живых существ".</li><li>"Человек" - сиюминутное "живое существо".</li></ul></li><li><ul><li>"Сократ" причастен "человеку" и "живому существу".</li><li>"Сократ" - полное пересечение классов "людей" и "живых существ".</li><li>"Сократ" - сейчас "живой" и "человек", а потом - нет.</li></ul></li><li><ul><li>"Бледный человек" причастен "бледности" и "человеку" одновременно.</li><li>"Бледный человек" - частичное пересечение классов "бледных" и "людей".</li><li>"Бледный человек" - сиюминутно "бледный" сиюминутный "человек".</li></ul>}

\footnotemark[6]{ Сейчас будет моя отсебятина, но, вроде, адекватная.}

\footnotemark[7]{ Я не могу это пересказать так, чтобы я сам понял.}

\end{document}

