\documentclass{article}

\usepackage[T2A]{fontenc}
\usepackage[utf8]{inputenc}
\usepackage[russian]{babel}

\linespread{1.1}
\setlength{\parskip}{1em}
\usepackage[left=1cm,right=1cm,top=1cm,bottom=2cm]{geometry}

\begin{document}

\footnotemark[1]{ Если ты можешь получить какой-то редкий опыт, например, залезть на гору - это не делает тебя мудрым. Это делает тебя сильным, смелым - но не мудрым.}

\footnotemark[2]{ Подробнее о том, почему "благо - то есть цель существования - это тоже причина" написало в следующей главе.}

\footnotemark[3]{Есть мнение, что этот абзац не принадлежит Аристотелю. Мы не можем знать этого наверняка, но стоит обратить внимание на две вещи. Во-первые, на неравномерность распределения темы бога по "Метафизике". Во-вторых, на оторванность абзаца: тема бога возникает внезапно, цельным куском, из ниоткуда и в никуда; тогда как и до, и после у Аристотеля каждый абзац связан с соседними. Этот абзац можно выкинуть - и логика повествования не пострадает.}

\end{document}

