\documentclass[oneside, 17pt, dvipsnames]{extbook}

\title{
    Τὰ μετὰ τὰ φυσικά
    \\
    Βιβλία 1
}
\author{
    Aristotle
    \\
    IV century Before Christ
}
\date{
    Translated by sir William David Ross
    \\
    in 1934 year
}

\usepackage{xltxtra}
\usepackage{polyglossia}
\setdefaultlanguage{russian}
\setotherlanguages{english, greek}
\PolyglossiaSetup{russian}{indentfirst=true}
\PolyglossiaSetup{english}{indentfirst=true}
\PolyglossiaSetup{greek}{indentfirst=true}
\PolyglossiaSetup{latex}{indentfirst=true}
\usepackage{indentfirst} % отделять первую строку раздела абзацным отступом тоже

\usepackage{epigraph}
\setlength{\epigraphwidth}{1400pt}

% colors and fonts
\usepackage{xcolor}
\definecolor{bgColor}{RGB}{33,33,33}
\definecolor{fgColor}{RGB}{188,170,164}
\definecolor{maColor}{RGB}{139,195,74}
\setromanfont[Color=fgColor]{Liberation Sans}
\setsansfont{Liberation Sans}
\defaultfontfeatures{Ligatures={TeX},Renderer=Basic}
\pagecolor{bgColor}
\everymath{\color{maColor}}
\everydisplay{\color{fgColor}}

% geometry
\linespread{1.1}
\setlength{\parskip}{1em}
\setlength{\marginparpush}{30pt}
\usepackage[paperwidth=11in,paperheight=22in,landscape,left=1cm,right=1cm,top=1cm,bottom=2cm,outer=18cm,marginparwidth=15cm,marginparsep=1cm]{geometry}

% start a paragraph and a margin note at the same line
\newcommand{\alignedmarginpar}[1]{
  \leavevmode
  \marginpar{\footnotesize #1}
  \ignorespaces
}

\usepackage{hyperref}

\begin{document}

\maketitle
\tableofcontents

\section{Book 1}

All men by nature desire to know. An indication of this is the delight we take in our senses; for even apart from their usefulness they are loved for themselves; and above all others the sense of sight. For not only with a view to action, but even when we are not going to do anything, we prefer seeing (one might say) to everything else. The reason is that this, most of all the senses, makes us know and brings to light many differences between things.

By nature animals are born with the faculty of sensation, and from sensation memory is produced in some of them, though not in others. And therefore the former are more intelligent and apt at learning than those which cannot remember; those which are incapable of hearing sounds are intelligent though they cannot be taught, e.g. the bee, and any other race of animals that may be like it; and those which besides memory have this sense of hearing can be taught.

The animals other than man live by appearances and memories, and have but little of connected experience; but the human race lives also by art and reasonings. Now from memory experience is produced in men; for the several memories of the same thing produce finally the capacity for a single experience. And experience seems pretty much like science and art, but really science and art come to men through experience; for 'experience made art', as Polus says, 'but inexperience luck.' Now art arises when from many notions gained by experience one universal judgement about a class of objects is produced. For to have a judgement that when Callias was ill of this disease this did him good, and similarly in the case of Socrates and in many individual cases, is a matter of experience; but to judge that it has done good to all persons of a certain constitution, marked off in one class, when they were ill of this disease, e.g. to phlegmatic or bilious people when burning with fevers-this is a matter of art.

With a view to action experience seems in no respect inferior to art, and men of experience succeed even better than those who have theory without experience. (The reason is that experience is knowledge of individuals, art of universals, and actions and productions are all concerned with the individual; for the physician does not cure man, except in an incidental way, but Callias or Socrates or some other called by some such individual name, who happens to be a man. If, then, a man has the theory without the experience, and recognizes the universal but does not know the individual included in this, he will often fail to cure; for it is the individual that is to be cured.) But yet we think that knowledge and understanding belong to art rather than to experience, and we suppose artists to be wiser than men of experience (which implies that Wisdom depends in all cases rather on knowledge); and this because the former know the cause, but the latter do not. For men of experience know that the thing is so, but do not know why, while the others know the 'why' and the cause. Hence we think also that the masterworkers in each craft are more honourable and know in a truer sense and are wiser than the manual workers, because they know the causes of the things that are done (we think the manual workers are like certain lifeless things which act indeed, but act without knowing what they do, as fire burns,-but while the lifeless things perform each of their functions by a natural tendency, the labourers perform them through habit); thus we view them as being wiser not in virtue of being able to act, but of having the theory for themselves and knowing the causes. And in general it is a sign of the man who knows and of the man who does not know, that the former can teach, and therefore we think art more truly knowledge than experience is; for artists can teach, and men of mere experience cannot.

Again, we do not regard any of the senses as Wisdom; yet surely these give the most authoritative knowledge of particulars. But they do not tell us the 'why' of anything-e.g. why fire is hot; they only say that it is hot.

At first he who invented any art whatever that went beyond the common perceptions of man was naturally admired by men, not only because there was something useful in the inventions, but because he was thought wise and superior to the rest. But as more arts were invented, and some were directed to the necessities of life, others to recreation, the inventors of the latter were naturally always regarded as wiser than the inventors of the former, because their branches of knowledge did not aim at utility. Hence when all such inventions were already established, the sciences which do not aim at giving pleasure or at the necessities of life were discovered, and first in the places where men first began to have leisure. This is why the mathematical arts were founded in Egypt; for there the priestly caste was allowed to be at leisure.

We have said in the Ethics what the difference is between art and science and the other kindred faculties; but the point of our present discussion is this, that all men suppose what is called Wisdom to deal with the first causes and the principles of things; so that, as has been said before, the man of experience is thought to be wiser than the possessors of any sense-perception whatever, the artist wiser than the men of experience, the masterworker than the mechanic, and the theoretical kinds of knowledge to be more of the nature of Wisdom than the productive. Clearly then Wisdom is knowledge about certain principles and causes.




\newpage
\section{Book 2}

Since we are seeking this knowledge, we must inquire of what kind are the causes and the principles, the knowledge of which is Wisdom. If one were to take the notions we have about the wise man, this might perhaps make the answer more evident. We suppose first, then, that the wise man knows all things, as far as possible, although he has not knowledge of each of them in detail; secondly, that he who can learn things that are difficult, and not easy for man to know, is wise (sense-perception is common to all, and therefore easy and no mark of Wisdom); again, that he who is more exact and more capable of teaching the causes is wiser, in every branch of knowledge; and that of the sciences, also, that which is desirable on its own account and for the sake of knowing it is more of the nature of Wisdom than that which is desirable on account of its results, and the superior science is more of the nature of Wisdom than the ancillary; for the wise man must not be ordered but must order, and he must not obey another, but the less wise must obey him.

Such and so many are the notions, then, which we have about Wisdom and the wise. Now of these characteristics that of knowing all things must belong to him who has in the highest degree universal knowledge; for he knows in a sense all the instances that fall under the universal. And these things, the most universal, are on the whole the hardest for men to know; for they are farthest from the senses. And the most exact of the sciences are those which deal most with first principles; for those which involve fewer principles are more exact than those which involve additional principles, e.g. arithmetic than geometry. But the science which investigates causes is also instructive, in a higher degree, for the people who instruct us are those who tell the causes of each thing. And understanding and knowledge pursued for their own sake are found most in the knowledge of that which is most knowable (for he who chooses to know for the sake of knowing will choose most readily that which is most truly knowledge, and such is the knowledge of that which is most knowable); and the first principles and the causes are most knowable; for by reason of these, and from these, all other things come to be known, and not these by means of the things subordinate to them. And the science which knows to what end each thing must be done is the most authoritative of the sciences, and more authoritative than any ancillary science; and this end is the good of that thing, and in general the supreme good in the whole of nature. Judged by all the tests we have mentioned, then, the name in question falls to the same science; this must be a science that investigates the first principles and causes; for the good, i.e. the end, is one of the causes.

That it is not a science of production is clear even from the history of the earliest philosophers. For it is owing to their wonder that men both now begin and at first began to philosophize; they wondered originally at the obvious difficulties, then advanced little by little and stated difficulties about the greater matters, e.g. about the phenomena of the moon and those of the sun and of the stars, and about the genesis of the universe. And a man who is puzzled and wonders thinks himself ignorant (whence even the lover of myth is in a sense a lover of Wisdom, for the myth is composed of wonders); therefore since they philosophized order to escape from ignorance, evidently they were pursuing science in order to know, and not for any utilitarian end. And this is confirmed by the facts; for it was when almost all the necessities of life and the things that make for comfort and recreation had been secured, that such knowledge began to be sought. Evidently then we do not seek it for the sake of any other advantage; but as the man is free, we say, who exists for his own sake and not for another's, so we pursue this as the only free science, for it alone exists for its own sake.

Hence also the possession of it might be justly regarded as beyond human power; for in many ways human nature is in bondage, so that according to Simonides 'God alone can have this privilege', and it is unfitting that man should not be content to seek the knowledge that is suited to him. If, then, there is something in what the poets say, and jealousy is natural to the divine power, it would probably occur in this case above all, and all who excelled in this knowledge would be unfortunate. But the divine power cannot be jealous (nay, according to the proverb, 'bards tell a lie'), nor should any other science be thought more honourable than one of this sort. For the most divine science is also most honourable; and this science alone must be, in two ways, most divine. For the science which it would be most meet for God to have is a divine science, and so is any science that deals with divine objects; and this science alone has both these qualities; for (1) God is thought to be among the causes of all things and to be a first principle, and (2) such a science either God alone can have, or God above all others. All the sciences, indeed, are more necessary than this, but none is better.

Yet the acquisition of it must in a sense end in something which is the opposite of our original inquiries. For all men begin, as we said, by wondering that things are as they are, as they do about self-moving marionettes, or about the solstices or the incommensurability of the diagonal of a square with the side; for it seems wonderful to all who have not yet seen the reason, that there is a thing which cannot be measured even by the smallest unit. But we must end in the contrary and, according to the proverb, the better state, as is the case in these instances too when men learn the cause; for there is nothing which would surprise a geometer so much as if the diagonal turned out to be commensurable.

We have stated, then, what is the nature of the science we are searching for, and what is the mark which our search and our whole investigation must reach.





\newpage
\section{Book 3}

Evidently we have to acquire knowledge of the original causes (for we say we know each thing only when we think we recognize its first cause), and causes are spoken of in four senses. In one of these we mean the substance, i.e. the essence (for the 'why' is reducible finally to the definition, and the ultimate 'why' is a cause and principle); in another the matter or substratum, in a third the source of the change, and in a fourth the cause opposed to this, the purpose and the good (for this is the end of all generation and change). We have studied these causes sufficiently in our work on nature, but yet let us call to our aid those who have attacked the investigation of being and philosophized about reality before us. For obviously they too speak of certain principles and causes; to go over their views, then, will be of profit to the present inquiry, for we shall either find another kind of cause, or be more convinced of the correctness of those which we now maintain.

Of the first philosophers, then, most thought the principles which were of the nature of matter were the only principles of all things. That of which all things that are consist, the first from which they come to be, the last into which they are resolved (the substance remaining, but changing in its modifications), this they say is the element and this the principle of things, and therefore they think nothing is either generated or destroyed, since this sort of entity is always conserved, as we say Socrates neither comes to be absolutely when he comes to be beautiful or musical, nor ceases to be when loses these characteristics, because the substratum, Socrates himself remains. just so they say nothing else comes to be or ceases to be; for there must be some entity-either one or more than one-from which all other things come to be, it being conserved.

Yet they do not all agree as to the number and the nature of these principles. Thales, the founder of this type of philosophy, says the principle is water (for which reason he declared that the earth rests on water), getting the notion perhaps from seeing that the nutriment of all things is moist, and that heat itself is generated from the moist and kept alive by it (and that from which they come to be is a principle of all things). He got his notion from this fact, and from the fact that the seeds of all things have a moist nature, and that water is the origin of the nature of moist things.

Some think that even the ancients who lived long before the present generation, and first framed accounts of the gods, had a similar view of nature; for they made Ocean and Tethys the parents of creation, and described the oath of the gods as being by water, to which they give the name of Styx; for what is oldest is most honourable, and the most honourable thing is that by which one swears. It may perhaps be uncertain whether this opinion about nature is primitive and ancient, but Thales at any rate is said to have declared himself thus about the first cause. Hippo no one would think fit to include among these thinkers, because of the paltriness of his thought.

Anaximenes and Diogenes make air prior to water, and the most primary of the simple bodies, while Hippasus of Metapontium and Heraclitus of Ephesus say this of fire, and Empedocles says it of the four elements (adding a fourth-earth-to those which have been named); for these, he says, always remain and do not come to be, except that they come to be more or fewer, being aggregated into one and segregated out of one.

Anaxagoras of Clazomenae, who, though older than Empedocles, was later in his philosophical activity, says the principles are infinite in number; for he says almost all the things that are made of parts like themselves, in the manner of water or fire, are generated and destroyed in this way, only by aggregation and segregation, and are not in any other sense generated or destroyed, but remain eternally.

From these facts one might think that the only cause is the so-called material cause; but as men thus advanced, the very facts opened the way for them and joined in forcing them to investigate the subject. However true it may be that all generation and destruction proceed from some one or (for that matter) from more elements, why does this happen and what is the cause? For at least the substratum itself does not make itself change; e.g. neither the wood nor the bronze causes the change of either of them, nor does the wood manufacture a bed and the bronze a statue, but something else is the cause of the change. And to seek this is to seek the second cause, as we should say,-that from which comes the beginning of the movement. Now those who at the very beginning set themselves to this kind of inquiry, and said the substratum was one, were not at all dissatisfied with themselves; but some at least of those who maintain it to be one-as though defeated by this search for the second cause-say the one and nature as a whole is unchangeable not only in respect of generation and destruction (for this is a primitive belief, and all agreed in it), but also of all other change; and this view is peculiar to them. Of those who said the universe was one, then none succeeded in discovering a cause of this sort, except perhaps Parmenides, and he only inasmuch as he supposes that there is not only one but also in some sense two causes. But for those who make more elements it is more possible to state the second cause, e.g. for those who make hot and cold, or fire and earth, the elements; for they treat fire as having a nature which fits it to move things, and water and earth and such things they treat in the contrary way.

When these men and the principles of this kind had had their day, as the latter were found inadequate to generate the nature of things men were again forced by the truth itself, as we said, to inquire into the next kind of cause. For it is not likely either that fire or earth or any such element should be the reason why things manifest goodness and, beauty both in their being and in their coming to be, or that those thinkers should have supposed it was; nor again could it be right to entrust so great a matter to spontaneity and chance. When one man said, then, that reason was present-as in animals, so throughout nature-as the cause of order and of all arrangement, he seemed like a sober man in contrast with the random talk of his predecessors. We know that Anaxagoras certainly adopted these views, but Hermotimus of Clazomenae is credited with expressing them earlier. Those who thought thus stated that there is a principle of things which is at the same time the cause of beauty, and that sort of cause from which things acquire movement.





\newpage
\section{Book 4}

One might suspect that Hesiod was the first to look for such a thing-or some one else who put love or desire among existing things as a principle, as Parmenides, too, does; for he, in constructing the genesis of the universe, says:

"Love first of all the Gods she planned".

And Hesiod says:

"First of all things was chaos made, and then Broad-breasted earth... And love, 'mid all the gods pre-eminent".

Which implies that among existing things there must be from the first a cause which will move things and bring them together. How these thinkers should be arranged with regard to priority of discovery let us be allowed to decide later; but since the contraries of the various forms of good were also perceived to be present in nature-not only order and the beautiful, but also disorder and the ugly, and bad things in greater number than good, and ignoble things than beautiful-therefore another thinker introduced friendship and strife, each of the two the cause of one of these two sets of qualities. For if we were to follow out the view of Empedocles, and interpret it according to its meaning and not to its lisping expression, we should find that friendship is the cause of good things, and strife of bad. Therefore, if we said that Empedocles in a sense both mentions, and is the first to mention, the bad and the good as principles, we should perhaps be right, since the cause of all goods is the good itself.

These thinkers, as we say, evidently grasped, and to this extent, two of the causes which we distinguished in our work on nature-the matter and the source of the movement-vaguely, however, and with no clearness, but as untrained men behave in fights; for they go round their opponents and often strike fine blows, but they do not fight on scientific principles, and so too these thinkers do not seem to know what they say; for it is evident that, as a rule, they make no use of their causes except to a small extent. For Anaxagoras uses reason as a deus ex machina for the making of the world, and when he is at a loss to tell from what cause something necessarily is, then he drags reason in, but in all other cases ascribes events to anything rather than to reason. And Empedocles, though he uses the causes to a greater extent than this, neither does so sufficiently nor attains consistency in their use. At least, in many cases he makes love segregate things, and strife aggregate them. For whenever the universe is dissolved into its elements by strife, fire is aggregated into one, and so is each of the other elements; but whenever again under the influence of love they come together into one, the parts must again be segregated out of each element.

Empedocles, then, in contrast with his precessors, was the first to introduce the dividing of this cause, not positing one source of movement, but different and contrary sources. Again, he was the first to speak of four material elements; yet he does not use four, but treats them as two only; he treats fire by itself, and its opposite-earth, air, and water-as one kind of thing. We may learn this by study of his verses.

This philosopher then, as we say, has spoken of the principles in this way, and made them of this number. Leucippus and his associate Democritus say that the full and the empty are the elements, calling the one being and the other non-being-the full and solid being being, the empty non-being (whence they say being no more is than non-being, because the solid no more is than the empty); and they make these the material causes of things. And as those who make the underlying substance one generate all other things by its modifications, supposing the rare and the dense to be the sources of the modifications, in the same way these philosophers say the differences in the elements are the causes of all other qualities. These differences, they say, are three-shape and order and position. For they say the real is differentiated only by 'rhythm and 'inter-contact' and 'turning'; and of these rhythm is shape, inter-contact is order, and turning is position; for A differs from N in shape, AN from NA in order, M from W in position. The question of movement-whence or how it is to belong to things-these thinkers, like the others, lazily neglected.

Regarding the two causes, then, as we say, the inquiry seems to have been pushed thus far by the early philosophers.





\newpage
\section{Book 5}

Contemporaneously with these philosophers and before them, the so-called Pythagoreans, who were the first to take up mathematics, not only advanced this study, but also having been brought up in it they thought its principles were the principles of all things. Since of these principles numbers are by nature the first, and in numbers they seemed to see many resemblances to the things that exist and come into being-more than in fire and earth and water (such and such a modification of numbers being justice, another being soul and reason, another being opportunity-and similarly almost all other things being numerically expressible); since, again, they saw that the modifications and the ratios of the musical scales were expressible in numbers;-since, then, all other things seemed in their whole nature to be modelled on numbers, and numbers seemed to be the first things in the whole of nature, they supposed the elements of numbers to be the elements of all things, and the whole heaven to be a musical scale and a number. And all the properties of numbers and scales which they could show to agree with the attributes and parts and the whole arrangement of the heavens, they collected and fitted into their scheme; and if there was a gap anywhere, they readily made additions so as to make their whole theory coherent. E.g. as the number 10 is thought to be perfect and to comprise the whole nature of numbers, they say that the bodies which move through the heavens are ten, but as the visible bodies are only nine, to meet this they invent a tenth--the 'counter-earth'. We have discussed these matters more exactly elsewhere.

But the object of our review is that we may learn from these philosophers also what they suppose to be the principles and how these fall under the causes we have named. Evidently, then, these thinkers also consider that number is the principle both as matter for things and as forming both their modifications and their permanent states, and hold that the elements of number are the even and the odd, and that of these the latter is limited, and the former unlimited; and that the One proceeds from both of these (for it is both even and odd), and number from the One; and that the whole heaven, as has been said, is numbers.

Other members of this same school say there are ten principles, which they arrange in two columns of cognates-limit and unlimited, odd and even, one and plurality, right and left, male and female, resting and moving, straight and curved, light and darkness, good and bad, square and oblong. In this way Alcmaeon of Croton seems also to have conceived the matter, and either he got this view from them or they got it from him; for he expressed himself similarly to them. For he says most human affairs go in pairs, meaning not definite contrarieties such as the Pythagoreans speak of, but any chance contrarieties, e.g. white and black, sweet and bitter, good and bad, great and small. He threw out indefinite suggestions about the other contrarieties, but the Pythagoreans declared both how many and which their contraricties are.

From both these schools, then, we can learn this much, that the contraries are the principles of things; and how many these principles are and which they are, we can learn from one of the two schools. But how these principles can be brought together under the causes we have named has not been clearly and articulately stated by them; they seem, however, to range the elements under the head of matter; for out of these as immanent parts they say substance is composed and moulded.

From these facts we may sufficiently perceive the meaning of the ancients who said the elements of nature were more than one; but there are some who spoke of the universe as if it were one entity, though they were not all alike either in the excellence of their statement or in its conformity to the facts of nature. The discussion of them is in no way appropriate to our present investigation of causes, for. they do not, like some of the natural philosophers, assume being to be one and yet generate it out of the one as out of matter, but they speak in another way; those others add change, since they generate the universe, but these thinkers say the universe is unchangeable. Yet this much is germane to the present inquiry: Parmenides seems to fasten on that which is one in definition, Melissus on that which is one in matter, for which reason the former says that it is limited, the latter that it is unlimited; while Xenophanes, the first of these partisans of the One (for Parmenides is said to have been his pupil), gave no clear statement, nor does he seem to have grasped the nature of either of these causes, but with reference to the whole material universe he says the One is God. Now these thinkers, as we said, must be neglected for the purposes of the present inquiry-two of them entirely, as being a little too naive, viz. Xenophanes and Melissus; but Parmenides seems in places to speak with more insight. For, claiming that, besides the existent, nothing non-existent exists, he thinks that of necessity one thing exists, viz. the existent and nothing else (on this we have spoken more clearly in our work on nature), but being forced to follow the observed facts, and supposing the existence of that which is one in definition, but more than one according to our sensations, he now posits two causes and two principles, calling them hot and cold, i.e. fire and earth; and of these he ranges the hot with the existent, and the other with the non-existent.

From what has been said, then, and from the wise men who have now sat in council with us, we have got thus much-on the one hand from the earliest philosophers, who regard the first principle as corporeal (for water and fire and such things are bodies), and of whom some suppose that there is one corporeal principle, others that there are more than one, but both put these under the head of matter; and on the other hand from some who posit both this cause and besides this the source of movement, which we have got from some as single and from others as twofold.

Down to the Italian school, then, and apart from it, philosophers have treated these subjects rather obscurely, except that, as we said, they have in fact used two kinds of cause, and one of these-the source of movement-some treat as one and others as two. But the Pythagoreans have said in the same way that there are two principles, but added this much, which is peculiar to them, that they thought that finitude and infinity were not attributes of certain other things, e.g. of fire or earth or anything else of this kind, but that infinity itself and unity itself were the substance of the things of which they are predicated. This is why number was the substance of all things. On this subject, then, they expressed themselves thus; and regarding the question of essence they began to make statements and definitions, but treated the matter too simply. For they both defined superficially and thought that the first subject of which a given definition was predicable was the substance of the thing defined, as if one supposed that 'double' and '2' were the same, because 2 is the first thing of which 'double' is predicable. But surely to be double and to be 2 are not the same; if they are, one thing will be many-a consequence which they actually drew. From the earlier philosophers, then, and from their successors we can learn thus much.





\newpage
\section{Book 6}

After the systems we have named came the philosophy of Plato, which in most respects followed these thinkers, but had pecullarities that distinguished it from the philosophy of the Italians. For, having in his youth first become familiar with Cratylus and with the Heraclitean doctrines (that all sensible things are ever in a state of flux and there is no knowledge about them), these views he held even in later years. Socrates, however, was busying himself about ethical matters and neglecting the world of nature as a whole but seeking the universal in these ethical matters, and fixed thought for the first time on definitions; Plato accepted his teaching, but held that the problem applied not to sensible things but to entities of another kind-for this reason, that the common definition could not be a definition of any sensible thing, as they were always changing. Things of this other sort, then, he called Ideas, and sensible things, he said, were all named after these, and in virtue of a relation to these; for the many existed by participation in the Ideas that have the same name as they. Only the name 'participation' was new; for the Pythagoreans say that things exist by 'imitation' of numbers, and Plato says they exist by participation, changing the name. But what the participation or the imitation of the Forms could be they left an open question.

Further, besides sensible things and Forms he says there are the objects of mathematics, which occupy an intermediate position, differing from sensible things in being eternal and unchangeable, from Forms in that there are many alike, while the Form itself is in each case unique.

Since the Forms were the causes of all other things, he thought their elements were the elements of all things. As matter, the great and the small were principles; as essential reality, the One; for from the great and the small, by participation in the One, come the Numbers.

But he agreed with the Pythagoreans in saying that the One is substance and not a predicate of something else; and in saying that the Numbers are the causes of the reality of other things he agreed with them; but positing a dyad and constructing the infinite out of great and small, instead of treating the infinite as one, is peculiar to him; and so is his view that the Numbers exist apart from sensible things, while they say that the things themselves are Numbers, and do not place the objects of mathematics between Forms and sensible things. His divergence from the Pythagoreans in making the One and the Numbers separate from things, and his introduction of the Forms, were due to his inquiries in the region of definitions (for the earlier thinkers had no tincture of dialectic), and his making the other entity besides the One a dyad was due to the belief that the numbers, except those which were prime, could be neatly produced out of the dyad as out of some plastic material. Yet what happens is the contrary; the theory is not a reasonable one. For they make many things out of the matter, and the form generates only once, but what we observe is that one table is made from one matter, while the man who applies the form, though he is one, makes many tables. And the relation of the male to the female is similar; for the latter is impregnated by one copulation, but the male impregnates many females; yet these are analogues of those first principles.

Plato, then, declared himself thus on the points in question; it is evident from what has been said that he has used only two causes, that of the essence and the material cause (for the Forms are the causes of the essence of all other things, and the One is the cause of the essence of the Forms); and it is evident what the underlying matter is, of which the Forms are predicated in the case of sensible things, and the One in the case of Forms, viz. that this is a dyad, the great and the small. Further, he has assigned the cause of good and that of evil to the elements, one to each of the two, as we say some of his predecessors sought to do, e.g. Empedocles and Anaxagoras.





\newpage
\section{Book 7}

Our review of those who have spoken about first principles and reality and of the way in which they have spoken, has been concise and summary; but yet we have learnt this much from them, that of those who speak about 'principle' and 'cause' no one has mentioned any principle except those which have been distinguished in our work on nature, but all evidently have some inkling of them, though only vaguely. For some speak of the first principle as matter, whether they suppose one or more first principles, and whether they suppose this to be a body or to be incorporeal; e.g. Plato spoke of the great and the small, the Italians of the infinite, Empedocles of fire, earth, water, and air, Anaxagoras of the infinity of things composed of similar parts. These, then, have all had a notion of this kind of cause, and so have all who speak of air or fire or water, or something denser than fire and rarer than air; for some have said the prime element is of this kind.

These thinkers grasped this cause only; but certain others have mentioned the source of movement, e.g. those who make friendship and strife, or reason, or love, a principle.

The essence, i.e. the substantial reality, no one has expressed distinctly. It is hinted at chiefly by those who believe in the Forms; for they do not suppose either that the Forms are the matter of sensible things, and the One the matter of the Forms, or that they are the source of movement (for they say these are causes rather of immobility and of being at rest), but they furnish the Forms as the essence of every other thing, and the One as the essence of the Forms.

That for whose sake actions and changes and movements take place, they assert to be a cause in a way, but not in this way, i.e. not in the way in which it is its nature to be a cause. For those who speak of reason or friendship class these causes as goods; they do not speak, however, as if anything that exists either existed or came into being for the sake of these, but as if movements started from these. In the same way those who say the One or the existent is the good, say that it is the cause of substance, but not that substance either is or comes to be for the sake of this. Therefore it turns out that in a sense they both say and do not say the good is a cause; for they do not call it a cause qua good but only incidentally.

All these thinkers then, as they cannot pitch on another cause, seem to testify that we have determined rightly both how many and of what sort the causes are. Besides this it is plain that when the causes are being looked for, either all four must be sought thus or they must be sought in one of these four ways. Let us next discuss the possible difficulties with regard to the way in which each of these thinkers has spoken, and with regard to his situation relatively to the first principles.





\newpage
\section{Book 8}

Those, then, who say the universe is one and posit one kind of thing as matter, and as corporeal matter which has spatial magnitude, evidently go astray in many ways. For they posit the elements of bodies only, not of incorporeal things, though there are also incorporeal things. And in trying to state the causes of generation and destruction, and in giving a physical account of all things, they do away with the cause of movement. Further, they err in not positing the substance, i.e. the essence, as the cause of anything, and besides this in lightly calling any of the simple bodies except earth the first principle, without inquiring how they are produced out of one anothers-I mean fire, water, earth, and air. For some things are produced out of each other by combination, others by separation, and this makes the greatest difference to their priority and posteriority. For (1) in a way the property of being most elementary of all would seem to belong to the first thing from which they are produced by combination, and this property would belong to the most fine-grained and subtle of bodies. For this reason those who make fire the principle would be most in agreement with this argument. But each of the other thinkers agrees that the element of corporeal things is of this sort. At least none of those who named one element claimed that earth was the element, evidently because of the coarseness of its grain. (Of the other three elements each has found some judge on its side; for some maintain that fire, others that water, others that air is the element. Yet why, after all, do they not name earth also, as most men do? For people say all things are earth Hesiod says earth was produced first of corporeal things; so primitive and popular has the opinion been.) According to this argument, then, no one would be right who either says the first principle is any of the elements other than fire, or supposes it to be denser than air but rarer than water. But (2) if that which is later in generation is prior in nature, and that which is concocted and compounded is later in generation, the contrary of what we have been saying must be true,-water must be prior to air, and earth to water.

So much, then, for those who posit one cause such as we mentioned; but the same is true if one supposes more of these, as Empedocles says matter of things is four bodies. For he too is confronted by consequences some of which are the same as have been mentioned, while others are peculiar to him. For we see these bodies produced from one another, which implies that the same body does not always remain fire or earth (we have spoken about this in our works on nature); and regarding the cause of movement and the question whether we must posit one or two, he must be thought to have spoken neither correctly nor altogether plausibly. And in general, change of quality is necessarily done away with for those who speak thus, for on their view cold will not come from hot nor hot from cold. For if it did there would be something that accepted the contraries themselves, and there would be some one entity that became fire and water, which Empedocles denies.

As regards Anaxagoras, if one were to suppose that he said there were two elements, the supposition would accord thoroughly with an argument which Anaxagoras himself did not state articulately, but which he must have accepted if any one had led him on to it. True, to say that in the beginning all things were mixed is absurd both on other grounds and because it follows that they must have existed before in an unmixed form, and because nature does not allow any chance thing to be mixed with any chance thing, and also because on this view modifications and accidents could be separated from substances (for the same things which are mixed can be separated); yet if one were to follow him up, piecing together what he means, he would perhaps be seen to be somewhat modern in his views. For when nothing was separated out, evidently nothing could be truly asserted of the substance that then existed. I mean, e.g. that it was neither white nor black, nor grey nor any other colour, but of necessity colourless; for if it had been coloured, it would have had one of these colours. And similarly, by this same argument, it was flavourless, nor had it any similar attribute; for it could not be either of any quality or of any size, nor could it be any definite kind of thing. For if it were, one of the particular forms would have belonged to it, and this is impossible, since all were mixed together; for the particular form would necessarily have been already separated out, but he all were mixed except reason, and this alone was unmixed and pure. From this it follows, then, that he must say the principles are the One (for this is simple and unmixed) and the Other, which is of such a nature as we suppose the indefinite to be before it is defined and partakes of some form. Therefore, while expressing himself neither rightly nor clearly, he means something like what the later thinkers say and what is now more clearly seen to be the case.

But these thinkers are, after all, at home only in arguments about generation and destruction and movement; for it is practically only of this sort of substance that they seek the principles and the causes. But those who extend their vision to all things that exist, and of existing things suppose some to be perceptible and others not perceptible, evidently study both classes, which is all the more reason why one should devote some time to seeing what is good in their views and what bad from the standpoint of the inquiry we have now before us.

The 'Pythagoreans' treat of principles and elements stranger than those of the physical philosophers (the reason is that they got the principles from non-sensible things, for the objects of mathematics, except those of astronomy, are of the class of things without movement); yet their discussions and investigations are all about nature; for they generate the heavens, and with regard to their parts and attributes and functions they observe the phenomena, and use up the principles and the causes in explaining these, which implies that they agree with the others, the physical philosophers, that the real is just all that which is perceptible and contained by the so-called 'heavens'. But the causes and the principles which they mention are, as we said, sufficient to act as steps even up to the higher realms of reality, and are more suited to these than to theories about nature. They do not tell us at all, however, how there can be movement if limit and unlimited and odd and even are the only things assumed, or how without movement and change there can be generation and destruction, or the bodies that move through the heavens can do what they do.

Further, if one either granted them that spatial magnitude consists of these elements, or this were proved, still how would some bodies be light and others have weight? To judge from what they assume and maintain they are speaking no more of mathematical bodies than of perceptible; hence they have said nothing whatever about fire or earth or the other bodies of this sort, I suppose because they have nothing to say which applies peculiarly to perceptible things.

Further, how are we to combine the beliefs that the attributes of number, and number itself, are causes of what exists and happens in the heavens both from the beginning and now, and that there is no other number than this number out of which the world is composed? When in one particular region they place opinion and opportunity, and, a little above or below, injustice and decision or mixture, and allege, as proof, that each of these is a number, and that there happens to be already in this place a plurality of the extended bodies composed of numbers, because these attributes of number attach to the various places,-this being so, is this number, which we must suppose each of these abstractions to be, the same number which is exhibited in the material universe, or is it another than this? Plato says it is different; yet even he thinks that both these bodies and their causes are numbers, but that the intelligible numbers are causes, while the others are sensible.





\newpage
\section{Book 9}

Let us leave the Pythagoreans for the present; for it is enough to have touched on them as much as we have done. But as for those who posit the Ideas as causes, firstly, in seeking to grasp the causes of the things around us, they introduced others equal in number to these, as if a man who wanted to count things thought he would not be able to do it while they were few, but tried to count them when he had added to their number. For the Forms are practically equal to-or not fewer than-the things, in trying to explain which these thinkers proceeded from them to the Forms. For to each thing there answers an entity which has the same name and exists apart from the substances, and so also in the case of all other groups there is a one over many, whether the many are in this world or are eternal.

Further, of the ways in which we prove that the Forms exist, none is convincing; for from some no inference necessarily follows, and from some arise Forms even of things of which we think there are no Forms. For according to the arguments from the existence of the sciences there will be Forms of all things of which there are sciences and according to the 'one over many' argument there will be Forms even of negations, and according to the argument that there is an object for thought even when the thing has perished, there will be Forms of perishable things; for we have an image of these. Further, of the more accurate arguments, some lead to Ideas of relations, of which we say there is no independent class, and others introduce the 'third man'.

And in general the arguments for the Forms destroy the things for whose existence we are more zealous than for the existence of the Ideas; for it follows that not the dyad but number is first, i.e. that the relative is prior to the absolute,-besides all the other points on which certain people by following out the opinions held about the Ideas have come into conflict with the principles of the theory.

Further, according to the assumption on which our belief in the Ideas rests, there will be Forms not only of substances but also of many other things (for the concept is single not only in the case of substances but also in the other cases, and there are sciences not only of substance but also of other things, and a thousand other such difficulties confront them). But according to the necessities of the case and the opinions held about the Forms, if Forms can be shared in there must be Ideas of substances only. For they are not shared in incidentally, but a thing must share in its Form as in something not predicated of a subject (by 'being shared in incidentally' I mean that e.g. if a thing shares in 'double itself', it shares also in 'eternal', but incidentally; for 'eternal' happens to be predicable of the 'double'). Therefore the Forms will be substance; but the same terms indicate substance in this and in the ideal world (or what will be the meaning of saying that there is something apart from the particulars-the one over many?). And if the Ideas and the particulars that share in them have the same form, there will be something common to these; for why should '2' be one and the same in the perishable 2's or in those which are many but eternal, and not the same in the '2' itself' as in the particular 2? But if they have not the same form, they must have only the name in common, and it is as if one were to call both Callias and a wooden image a 'man', without observing any community between them.

Above all one might discuss the question what on earth the Forms contribute to sensible things, either to those that are eternal or to those that come into being and cease to be. For they cause neither movement nor any change in them. But again they help in no wise either towards the knowledge of the other things (for they are not even the substance of these, else they would have been in them), or towards their being, if they are not in the particulars which share in them; though if they were, they might be thought to be causes, as white causes whiteness in a white object by entering into its composition. But this argument, which first Anaxagoras and later Eudoxus and certain others used, is very easily upset; for it is not difficult to collect many insuperable objections to such a view.

But, further, all other things cannot come from the Forms in any of the usual senses of 'from'. And to say that they are patterns and the other things share in them is to use empty words and poetical metaphors. For what is it that works, looking to the Ideas? And anything can either be, or become, like another without being copied from it, so that whether Socrates or not a man Socrates like might come to be; and evidently this might be so even if Socrates were eternal. And there will be several patterns of the same thing, and therefore several Forms; e.g. 'animal' and 'two-footed' and also 'man himself' will be Forms of man. Again, the Forms are patterns not only sensible things, but of Forms themselves also; i.e. the genus, as genus of various species, will be so; therefore the same thing will be pattern and copy.

Again, it would seem impossible that the substance and that of which it is the substance should exist apart; how, therefore, could the Ideas, being the substances of things, exist apart? In the Phaedo' the case is stated in this way-that the Forms are causes both of being and of becoming; yet when the Forms exist, still the things that share in them do not come into being, unless there is something to originate movement; and many other things come into being (e.g. a house or a ring) of which we say there are no Forms. Clearly, therefore, even the other things can both be and come into being owing to such causes as produce the things just mentioned.

Again, if the Forms are numbers, how can they be causes? Is it because existing things are other numbers, e.g. one number is man, another is Socrates, another Callias? Why then are the one set of numbers causes of the other set? It will not make any difference even if the former are eternal and the latter are not. But if it is because things in this sensible world (e.g. harmony) are ratios of numbers, evidently the things between which they are ratios are some one class of things. If, then, this--the matter--is some definite thing, evidently the numbers themselves too will be ratios of something to something else. E.g. if Callias is a numerical ratio between fire and earth and water and air, his Idea also will be a number of certain other underlying things; and man himself, whether it is a number in a sense or not, will still be a numerical ratio of certain things and not a number proper, nor will it be a of number merely because it is a numerical ratio.

Again, from many numbers one number is produced, but how can one Form come from many Forms? And if the number comes not from the many numbers themselves but from the units in them, e.g. in 10,000, how is it with the units? If they are specifically alike, numerous absurdities will follow, and also if they are not alike (neither the units in one number being themselves like one another nor those in other numbers being all like to all); for in what will they differ, as they are without quality? This is not a plausible view, nor is it consistent with our thought on the matter.

Further, they must set up a second kind of number (with which arithmetic deals), and all the objects which are called 'intermediate' by some thinkers; and how do these exist or from what principles do they proceed? Or why must they be intermediate between the things in this sensible world and the things-themselves?

Further, the units in must each come from a prior but this is impossible.

Further, why is a number, when taken all together, one?

Again, besides what has been said, if the units are diverse the Platonists should have spoken like those who say there are four, or two, elements; for each of these thinkers gives the name of element not to that which is common, e.g. to body, but to fire and earth, whether there is something common to them, viz. body, or not. But in fact the Platonists speak as if the One were homogeneous like fire or water; and if this is so, the numbers will not be substances. Evidently, if there is a One itself and this is a first principle, 'one' is being used in more than one sense; for otherwise the theory is impossible.

When we wish to reduce substances to their principles, we state that lines come from the short and long (i.e. from a kind of small and great), and the plane from the broad and narrow, and body from the deep and shallow. Yet how then can either the plane contain a line, or the solid a line or a plane? For the broad and narrow is a different class from the deep and shallow. Therefore, just as number is not present in these, because the many and few are different from these, evidently no other of the higher classes will be present in the lower. But again the broad is not a genus which includes the deep, for then the solid would have been a species of plane. Further, from what principle will the presence of the points in the line be derived? Plato even used to object to this class of things as being a geometrical fiction. He gave the name of principle of the line-and this he often posited-to the indivisible lines. Yet these must have a limit; therefore the argument from which the existence of the line follows proves also the existence of the point.

In general, though philosophy seeks the cause of perceptible things, we have given this up (for we say nothing of the cause from which change takes its start), but while we fancy we are stating the substance of perceptible things, we assert the existence of a second class of substances, while our account of the way in which they are the substances of perceptible things is empty talk; for 'sharing', as we said before, means nothing.

Nor have the Forms any connexion with what we see to be the cause in the case of the arts, that for whose sake both all mind and the whole of nature are operative,-with this cause which we assert to be one of the first principles; but mathematics has come to be identical with philosophy for modern thinkers, though they say that it should be studied for the sake of other things. Further, one might suppose that the substance which according to them underlies as matter is too mathematical, and is a predicate and differentia of the substance, ie. of the matter, rather than matter itself; i.e. the great and the small are like the rare and the dense which the physical philosophers speak of, calling these the primary differentiae of the substratum; for these are a kind of excess and defect. And regarding movement, if the great and the small are to he movement, evidently the Forms will be moved; but if they are not to be movement, whence did movement come? The whole study of nature has been annihilated.

And what is thought to be easy-to show that all things are one-is not done; for what is proved by the method of setting out instances is not that all things are one but that there is a One itself,-if we grant all the assumptions. And not even this follows, if we do not grant that the universal is a genus; and this in some cases it cannot be.

Nor can it be explained either how the lines and planes and solids that come after the numbers exist or can exist, or what significance they have; for these can neither be Forms (for they are not numbers), nor the intermediates (for those are the objects of mathematics), nor the perishable things. This is evidently a distinct fourth class.

In general, if we search for the elements of existing things without distinguishing the many senses in which things are said to exist, we cannot find them, especially if the search for the elements of which things are made is conducted in this manner. For it is surely impossible to discover what 'acting' or 'being acted on', or 'the straight', is made of, but if elements can be discovered at all, it is only the elements of substances; therefore either to seek the elements of all existing things or to think one has them is incorrect.

And how could we learn the elements of all things? Evidently we cannot start by knowing anything before. For as he who is learning geometry, though he may know other things before, knows none of the things with which the science deals and about which he is to learn, so is it in all other cases. Therefore if there is a science of all things, such as some assert to exist, he who is learning this will know nothing before. Yet all learning is by means of premisses which are (either all or some of them) known before,-whether the learning be by demonstration or by definitions; for the elements of the definition must be known before and be familiar; and learning by induction proceeds similarly. But again, if the science were actually innate, it were strange that we are unaware of our possession of the greatest of sciences.

Again, how is one to come to know what all things are made of, and how is this to be made evident? This also affords a difficulty; for there might be a conflict of opinion, as there is about certain syllables; some say za is made out of s and d and a, while others say it is a distinct sound and none of those that are familiar.

Further, how could we know the objects of sense without having the sense in question? Yet we ought to, if the elements of which all things consist, as complex sounds consist of the clements proper to sound, are the same.




\newpage
\section{Book 10}

It is evident, then, even from what we have said before, that all men seem to seek the causes named in the Physics, and that we cannot name any beyond these; but they seek these vaguely; and though in a sense they have all been described before, in a sense they have not been described at all. For the earliest philosophy is, on all subjects, like one who lisps, since it is young and in its beginnings. For even Empedocles says bone exists by virtue of the ratio in it. Now this is the essence and the substance of the thing. But it is similarly necessary that flesh and each of the other tissues should be the ratio of its elements, or that not one of them should; for it is on account of this that both flesh and bone and everything else will exist, and not on account of the matter, which he names,-fire and earth and water and air. But while he would necessarily have agreed if another had said this, he has not said it clearly.

On these questions our views have been expressed before; but let us return to enumerate the difficulties that might be raised on these same points; for perhaps we may get from them some help towards our later difficulties.

\end{document}
