\documentclass{article}

\usepackage[T2A]{fontenc}
\usepackage[utf8]{inputenc}

\linespread{1.1}
\setlength{\parskip}{1em}
\usepackage[left=1cm,right=1cm,top=1cm,bottom=2cm]{geometry}

\begin{document}

It is evident, then, even from what we have said before, that all men seem to seek the causes named in the Physics, and that we cannot name any beyond these; but they seek these vaguely; and though in a sense they have all been described before, in a sense they have not been described at all. For the earliest philosophy is, on all subjects, like one who lisps, since it is young and in its beginnings. For even Empedocles says bone exists by virtue of the ratio in it. Now this is the essence and the substance of the thing. But it is similarly necessary that flesh and each of the other tissues should be the ratio of its elements, or that not one of them should; for it is on account of this that both flesh and bone and everything else will exist, and not on account of the matter, which he names,-fire and earth and water and air. But while he would necessarily have agreed if another had said this, he has not said it clearly.

On these questions our views have been expressed before; but let us return to enumerate the difficulties that might be raised on these same points; for perhaps we may get from them some help towards our later difficulties.

\end{document}

