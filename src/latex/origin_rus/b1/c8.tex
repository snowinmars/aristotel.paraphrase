\documentclass{article}

\usepackage[T2A]{fontenc}
\usepackage[utf8]{inputenc}
\usepackage[russian]{babel}

\linespread{1.1}
\setlength{\parskip}{1em}
\usepackage[left=1cm,right=1cm,top=1cm,bottom=2cm]{geometry}

\begin{document}

Те, которые признают единство вселенной и вводят единую материальную основу, считая таковую телесной и протяженной явным образом ошибаются во многих отношениях. В самом деле, они устанавливают элементы только для тел, а для бестелесных вещей --- нет, в то время как есть и вещи бестелесные. Точно так же, начиная указывать причины для возникновения и уничтожения и рассматривая все вещи с натуралистической точки зренияа , они упраздняют причину движения. Далее (они погрешают), не выставляя сущность причиною чего-либо, равно как и суть (вещи), и, кроме того, опрометчиво объявляют началом любое из простых тел, за исключением земли,
\footnotemark[1]
не подвергнув рассмотрению, как совершается их возникновение друг из друга [я говорю здесь об огне, воде, земло и воздухе] . В самом деле, однп вещи происходят друг из друга через соединение, другие --- через разделение, а это раз личие имеет самое большое значение для вопроса, что --- раньше и что --- позже. При одной точке зрения наиболее элементарным могло бы показаться то, из чего как из первого вещи возникают через соединение* а таковым было бы тело с наиболее мелкими и тонкими частями . Поэтому все, кто признает началом огонь, стоят, можно» сказать, в наибольшем согласии с этой точкой зрения. И так же смотрит на элементарную основу тел и каждый из остальных (философов). По крайней море никто* из числа тех, кто выступал позже и утверждал единство (первоосновы), не выставил требования, чтобы земля была элементом --- очевидно, вследствие того, что (у нее) крупные части; а из трех (других) элементов каждый получил какого-нибудь заступника: одни ставят на это место огонь, другие --- воду, третьи --- воздух. И однако же почему они не указывают и землю, как это делает большинство людей? Ведь люди» говорят, что все есть земля, да и Геспод указывает, что земля произошла раньше нсех тел : настолько древним и популярным оказывается это мнение. Итак, если стоять па этой точке зрения, то было бы неправильно признавать (началом) какойлибо из этих элементов, кроме огня, и точно так же --- считать, что такое начало плотнее воздуха, но разреженнее воды. Если же то, что позднее по происхождению, раньше по природе, а переработанное и смешанное --- но происхождению позднее, то получается обратный порядок: вода будет раньше воздуха, а земля --- раньше воды.

Относительно тех, которые устанавливают одну такую причину, как мы указали, ограничимся оказанным. То же получается и в том случае, если кто устанавливает несколько таких причин, --- как, например, Эмпедокл указывает, что материю образуют четыре тела: и у него также должны получиться частью то же самые, частью специальные затруднения. В самом деле, мы видим, что элементы возникают друг из друга, так что огонь и земля но всегда остаются тем же самым телом (об этом сказано в сочинениях о природе
\footnotemark[2]
); а относительно причины движущихся тел, принимать ли одну такую причину или две, об этом, надо считать, ,у него) совсем не сказано сколько-нибудь правильно или обоснованно. Вообще у тех, кто говорит таким образом, необходимо упраздняется качественное изменение; ибо не может (у них) получиться пи холодное из теплого, ни теплое из холодного. В самом деле, (тогда) чему-нибудь должны были пы принадлежать сами эти противоположные свойства и должно было бы существовать какое-нпбудь одно вещество, которое. становится огнем и водой, --- а он отрицает это.

Что касается (теперь) Анаксагора, то ест сказать, что он принимает два элемента, это наиболее соответствовало бы правильному ходу мысли, причем сам он его, правда, не продумал, но необходимо после довал бы за теми , кто стал бы указывать ему (надлежащий) путь. В самом деле, (я уже не говорю о том, что) (по ряду причин) нелепо утверждать изначальное смешение всех вещей, --- и потому, что онп в таком случае должны были бы ранее существовать в несмешанном виде; и потому, что от природы не свойственно смешиваться чему попало с чем попало; а кроме того и потому, что отдельные состояния и привходящие (случайные) свойства отделялись бы (в таком случае) от субстанций (одно и то же ведь подвергается смешению и отделению): но при всем том, если бы последовать (за ним), анализируя вместе (с ним) то, что он хочет сказать, то его слова произвели бы более современное впечатление. В то время, когда ничего не было выделено, об этой субстанции, очевидно, ничего нельзя было правильно сказать; я имею в виду, например, что она не была ни белого, ни черного, ни серого или иного цвета, но необходимо была бесцветной, --- иначе у нее был бы какой-нибудь из этих цветов. И подобным же образом она была без вкуса на этом же самом основании, и у нее не было никакого другого из подобных (свойств). Ибо для нее невозможно нметь ни качественную определенность, ни количественную, ни определенность по существу:
\footnotemark[3]
в таком случае у нее была бы какая-нибудь из так называемых частичных форм , а это невозможно, раз всо находилось в смешении; тогда уже произошло бы выделение, а между тем он утверждает, что все было смешано, кроме ума, и лишь он один --- несмешап и чист. Из сказанного для Анаксагора получается, что он в качестве начал указывает единое (оно ведь является простым и несмешанным) и «иное» (это последнее --- в том смысле), как мы
\footnotemark[4]
нрнзнаем неопределенное --- до того как оно получило определенность и приняло какую-ниоудь форму. Таким образом он говорит неправильно и неясно, но в своих намерениях приближается к мыслителям, вы ступающим позднее, и к более принятым теперь взглядама.

Эти философы, однако, близко занялись только рассуждениями о возникновении, уничтожении и движении: и начала и причины они исследуют почти исключительно в отношении такого рода сущности.
\footnotemark[5]
Те же, которые подвергают рассмотрению всю совокупность бытия, а в области бытия различают, с одной стороны, чувственные, с другой --- нечувственные вещи, (такие мыслители), очевидно, производят исследование обоих (этих) родов, и поэтому можно бы более обстоятельно заняться ими, (определяя), что сказано у них удачно или неудачно для выяснения стоящих теперь перед нами вопросов.

Что касается так называемых пифагорейцев, то они пользуются более необычными началами и элементами, нежели философы природы (причина здесь --- в том, что они к началам этим пришли не от чувственных вещей; ибо математические предметы чужды движению, за исключением тех, которые относятся к астрономии ); но прп этом все свои 'рассуждения и занятия они сосредоточивают на природе. В самом деле, они построяют небо и прослеживают то, что получается для его частей, состояний и действий, и на это используют свои начала и причины, как бы соглашаясь с другими натурфилософами, что бытием является (лишь) то, что воспринимается чувствами и что объемлет так называемое небо. Однако же, как мы сказали, причины и начала, которые они указывают, достаточны для того, чтобы подняться и в более высокую область бытия, и более подходят (для этого), нежели для рассуждений о природе. Но, с другой стороны, откуда получится движение, когда в основе лежат только предел и беспредельное, нечетное и четное, --- об ;)том они ничего не говорят, и вместе с тем (не указывают) --- как возможно, чтобы без движения и изменения происходили возникновение и уничтожение, или действия несущихся по небу (тел).

Далее, если бы и признать вместе с ними, что из этих начал (т. е. предела и беспредельного) образуется величина, или если бы было доказано это, --- все же каким образом получится, что одни тела --- легкие, а другие --- имеют тяжесть? В самом деле, исходя от тех начал, которые они кладут в основу и указывают, они не в меньшей мере берутся давать разъяснения относительно чувственных тел, чем относительно математических; в соответствии с тем об огне, земле и других таких телах у них совсем ничего не сказано, думаю --- потому, что в отношении чувственных вещей они никаких специальных указаний не давали.

Далее , как это понять, что свойства числа и (само) число являются причиной того, что и изначала и в настоящее время существует на небе и совершается в нем, а (вместе с тем) нет никакого другого числа, кроме того, из которого состоит вселенная? Если у них в такой-то части (мира) находится мнение и (в такой-то) удача, а немного выше или ниже --- несправедливость и отделение или смешение, причем в доказательство тому они приводят, что каждое из этих явлений есть число, а в данном месте оказывается уже (именно) это количество находящихся рядом (сосуществующих) (небесных) тел , вследствие чегоао указанные явления сопутствуют каждый раз соответственным местам; (если, таким образом, у них устанавливается тесная связь между отдельными явлениями и числами, господствующими в разных частях мира), --- то будем ли мы иметь здесь (по отношению к явлениям) тоже самое находящееся на небе число, про которое надлежит принять, что оно составляет каждое данное явление, или же здееь будет не это (мирообразующее) число, а другое? Платон (во всяком случае) говорит, что --- другое: правда, и он считает числами и вещи и причины вещей, но причинами он считает числа умопостигаемые, а те, которые отождествляются с вещами, воспринимаются чувствами.

\end{document}

