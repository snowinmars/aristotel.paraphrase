\documentclass{article}

\usepackage[T2A]{fontenc}
\usepackage[utf8]{inputenc}
\usepackage[russian]{babel}

\linespread{1.1}
\setlength{\parskip}{1em}
\usepackage[left=1cm,right=1cm,top=1cm,bottom=2cm]{geometry}

\begin{document}

После указанных философских учений появилось исследование Платона, в большинстве вопросов примыкающее к пифагорейцам, а в некоторых отношениях имеющее свои особенности по сравнению с философией италийцев. Смолоду сблизившись прежде всего с Кратилом и гераклитовскими учениями, по которым все чувственные вещи находятся в постоянном течении и знания об этих вещах не существует, он здесь и позже держался таких воззрений. А так как Сократ занимался исследованием этических вопросов, а относительно всей природы в целом его совсем не вел, в названной же области искал всеобщего и первый направил свою мысль на общие определения, то Платон, усвоивши взгляд Сократа, по указанной причине признал, что такие определения имеют своим предметом нечто другое, а не чувственные вещи; ибо нельзя дать общего определения для какой-нибудь из чувственных вещей, поскольку вещи эти постоянно изменяются. Идя указанным путем, он подобные реальности
\footnotemark[1]
назвал идеями, а что касается чувственных вещей, то об них (по его словам) речь всегда идет
\footnotemark[2]
отдельно от идей и (в то же время) в соответствии с ними; ибо все множество вещей существует в силу приобщения к одноименным (сущностям).
\footnotemark[3]
При этом, говоря о приобщении, он переменил только имя: пифагорейцы утверждают, что вещи существуют по подражанию числам, а Платон — что по приобщению [изменивши имя].
\footnotemark[4]
Но самое это приобщение или подражание идеям, что оно такое, — исследование этого вопроса было у них оставлено в стороне. Далее, помимо чувственных предметов и идей, он в промежутке устанавливает математические вещи, которые от чувственных предметов отличаются тем, что они вечные и неподвижные, а от идей — тем, что этих вещей имеется некоторое количество сходных друг с другом - сама же идея каждый раз только одна. И так как идеи являются причинами для всего остального, то их элементы он счел элементами всех вещей. Таким образом в качестве материи являются началами большое и малое, а в качестве сущности — единое; ибо идеи [они же числа]
\footnotemark[5]
получаются из большого и малого через приобщение (их) к единству. Что единое представляет собою сущность, а не носит наименование единого, будучи чем-либо другим, это Платон утверждал подобно пифагорейцам, и точно так же, как они
\footnotemark[6]
— что числа являются для всех остальных вещей причинами сущности (в них); а что он вместо неопределенного, как (чего-то) одного, ввел двоицу (пару) и составил неопределенное из большого и малого, это — его своеобразная черта; кроме того, он полагает числа отдельно от чувственных вещей, а они говорят, что числа, это — сами вещи, и математические объекты в промежутке между теми и другими не помещают. Установление единого и чисел отдельно от вещей, а не так, как у пифагорейцев, и введение идей произошло вследствие исследования в области понятий (более ранние философы к диалектике не были причастны), а двоица (пара) была принята за другую основу потому, что числа, за исключением первых
\footnotemark[7]
,легко выводились (рождались) из нее как из некоторой первичной массы. Однако же на самом деле происходит наоборот: то, что указывается здесь, не имеет хорошего основания. Эти философы из материи выводят множество, а идея (у них) рождает только один раз, между тем из одной материи, очевидно, получается один стол, а тот, который привносит идею
\footnotemark[8]
, будучи один, производит много их (столов). Подобным же образом относится и мужское начало к женскому: это последнее заполняется через одно соитие, а мужское заполняет много (самок); и однако же это — подобия тех начал.
\footnotemark[9]

Вот какие определения установил, таким образом, Платон по исследуемым (нами) вопросам. Из сказанного ясно, что он воспользовался только двумя причинами: причиною, определяющею суть вещи, и причиною из области материи (ибо идеи являются для всех остальных вещей причиною их сути, а для идей (таковою причиной является) единое); ясно также, что представляет собою материя, лежащая в основе (всего), — материя, которая получает определения через идеи при образовании чувственных вещей и (определения) через единое при образовании идей: эта материя есть двоица (пара), большое и малое. Кроме того, он связал с этими элементами
\footnotemark[10]
причину добра и причину зла, одну отнес к одному, другую — к другому, по образцу того, как, согласно нашим словам, искали ее
\footnotemark[11]
и некоторые из более ранних философов, например Эмпедокл и Анаксагор.


\end{document}

