\documentclass{article}

\usepackage[T2A]{fontenc}
\usepackage[utf8]{inputenc}
\usepackage[russian]{babel}

\linespread{1.1}
\setlength{\parskip}{1em}
\usepackage[left=1cm,right=1cm,top=1cm,bottom=2cm]{geometry}

\begin{document}

Одновременно с этими философами и {еще) раньше их так называемые пифа горейцы, занявшись математическими на уками, впервые двинули их вперед и, вос- питавшись на них, стали считать их начала  началами всех вещей. Но в области этих  наук числа занимают от природы первое место, а у чисел они усматривали, казалось им, много сходных черт1  с тем, что существует и происходит, --- больше, чем у огня, земли и воды, например такое-то свойство чисел есть справедливость, а такое-то душа и ум, другое --- удача, и можно сказать --- в каждом из остальных случаев точно так же. Кроме того, они видели в числах свойства и отношения8 , присущие гармоническим сочетаниям. Так как, следовательно, все остальное явным образом уподоблялось числам по всему своему существу, а числа занимали первое место во всей природе, элементы чисел они предположили элементами всех вещей и всю вселенную (признали) гармонией и числом. И все, что они могли в числах и гармонических сочетаниях показать согласующегося с состояниями и частями мира и со всем мировым устройством, это они сводили вместе и приспособляли (одно к другому); и если у них где-нибудь того или иного нехватало, они стремились (добавить это так), чтобы все построение8  находилось у них в сплошной связи. Так, например, ввиду того, что десятка (декада), как им представляется, есть нечто совершенное и вместила в себе всю природу чисел, то и несущихся по небу тел они считают десять, а так как видимых тел только девять, поэтому на десятом месте они помещают противоземлю.
\footnotemark[1]
В другом сочинении у нас об этом разъяснено точнее.
\footnotemark[2]
А ради чего мы касаемся этой области, так это --- для того, чтобы и у них установить, какие они полагают начала и как начала эти сводятся к указанным выше причинам. Во всяком случае и у них, повидимому, число принимается за начало и в качестве материи для вещей и в качестве (выражения для) их состояний и свойств6 , а элементами числа они считают чет и нечет, из коих первый является неопределенным, а второй определенным; единое состоит у них из того и другого, --- оно является и четным и нечетным, число (обра зуется) из единого, а (различные) числа, как было сказано, это --- вся вселенная.

Другие из этих же мыслителей принимают десять начал, идущих (каждый раз) в одном ряду --- предел и беспредельное, нечет и чет, единое и множество, правое и левое, мужское и женское, покоящееся и движущееся, прямое и кривое, свет и тьму, хорошее и дурное, четыреугольное и разностороннее.
\footnotemark[3]
Такого же мнения, повидимому, держался и Алкмеон Кротонец,
\footnotemark[4]
и либо он перенял это учение у тех мыслителей, либо они у него. И по времени ведь Алкмеон приходится на годы старости Пифагора8 , а высказался он подобно им. Он утверждает, что большинство свойств, с которыми имеют дело люди, составляют пары, но указывает противоположности --- не определенные, как те мыслители, а какие случится, например белое --- черное, сладкое --- горькое, хорошее --- дурное, большое --- малое. Он, таким образом, отбросил остальные вопросы, не давая (ближайших) определений, а пифагорейцы указали и сколько противоположностей и какие они.

И в том и в другом случае мы, следовательно, узнаем, что противоположности суть начала вещей; но сколько их --- узнаем у одних пифагорейцев, и также --- какие они. А как можно (принимаемые пифагорейцами начала) свести к указанным выше причинам, это у них
\footnotemark[5]
ясно не расчленено, но, повидимому, они помещают свои элементы в разряд материи; ибо по их словам из этих элементов, как из внутри находящихся9  частей, составлена и образована сущность.

На основании сказанного (можно) достаточно познакомиться с образом мыслей древних философов, устанавливавших для природы несколько элементов. Есть, однако, и такие, которые высказались о вселенной как об единой природе10 , но при этом вы сказались не одинаково --- ни в отношении удачностц сказанного, ни в отношении существа дела  и . Правда, в связи с теперешним рассмотрением причин говорить о них  совсем не приходится13  (они не поступают подобно некоторым натурфилософам13 , которые, предположив сущее единым, вместе с тем производят (вселенную) из единого, как из материи, но высказываются иначе, --- те мыслители добавляют (к единому) движение, по крайней мере когда производят вселенную, а эти оставляют его неподвижным). Но вот что во всяком случае имеет отношение к теперешнему исследованию. Парменпд, повидлмому, занимается единым, которое соответствует понятию,
\footnotemark[6]
Мелисс --- единым, которое соответствует материи. Поэтому один объявляет его ограниченным, другой --- неограниченным; а Ксенофан, который раньше их всех принял единство (говорят, что Парменид был его учеником), ничего не различил ясно и не коснулся ни той, ни другой природы14 , (указанной этими мыслителями)  15 , но, воззревши на небо в его целости, он заявляет, что единое, вот что такое бог. Этих мыслителей, как мы сказали, с точки зрения теперешнего исследования, надлежит оставить в стороне, двоих притом, именно Ксенофана и Мелисса, даже совсем --- так как они немного грубоваты; что же касается Парменида, то в его словах, повидимому, больше проницательности. Признавая, что небытио отдельно от сущего есть ничто, он считает, что по необходимости существует (только), одно, а именно --- сущее, и больше ничего (об этом мы яснее сказали в книгах о природе).
\footnotemark[7]
Однако же вынуждаемый сообразоваться с явлениями и признавая, что единое существует соответственно понятию, а множественность --- соответственно чувственному восприятию, он затем устанавливает две причины и два начала --- теплое н холодное, а именно говорит об огне и земле; причем из этих двух он к бытию относит теплое, а другое начало --- к небытию.

Вот все, что мы извлекли из сказанного ' ранее и у мудрецов, принявших уже участие в выяснении этого вопроса. Первые, нз них признавали10 , что начало является телесным (вода, огонь и току подобные ве щи суть тела), причем некоторые принимали, что телесное начало одно, а другие --- что таких начал несколько, но те и другие относили их в разряд материи; а были такие, которые принимали и эту причину и кроме нее --- ту, от которой исходит движение, причем эту последнюю некоторые (признавали) одну, а другие (указывали) (их) две.

Вплоть до италийских философов
\footnotemark[8]
и не считая их, все остальные высказались о началах в довольно скромной мере, но только, как мы сказали, воспользовались двумя причинами, и из них вторую --- ту, откуда движение --- некоторые устанавливают одну, а другие (вводят их) две. Что же касается пифагорейцев, то они таким же точно образом сказали, что есть два начала,
\footnotemark[9]
а прибавилось у них, как раз и составляя их отличительную черту, только то, что ограниченное, неограниченное и единое --- это, по их мнению, не свойства некоторых других физических реальностей, например огня или земли, или еще чего-нибудь в этом роде, но само неопределенное и само единое были (у них) сущностью того, о чем (то и другое) сказываются,
\footnotemark[10]
вследствие чего число и составляло (у них) сущность всех вещей. Вот каким образом высказались они по этому вопросу, и относительно сути (вещи)
\footnotemark[11]
они начали рассуждать н давать (ее) определение, но действовали слишком упрощенно. Определения их были поверхностны  21 , и то, к чему указанное (ими) определение подходило в первую очередь, это они и считали сущностью предмета, как если бы кто думал, что двойное и число два есть одно и то же, потому что двойное составляет в первую очередь свойство двух. Но ведь пожалуй что не одно и тоже --- быть двойным и быть двумя, а в противном случае одно (и то же) будет несколькими (разными) вещами,
\footnotemark[12]
как это получалось также и у них. --- Таким образом, вот все, что можно извлечь у более ранних философов и у других  22  (которые были после нпх).
\footnotemark[13]

\end{document}

