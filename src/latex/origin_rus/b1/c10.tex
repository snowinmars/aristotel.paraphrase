\documentclass{article}

\usepackage[T2A]{fontenc}
\usepackage[utf8]{inputenc}
\usepackage[russian]{babel}

\linespread{1.1}
\setlength{\parskip}{1em}
\usepackage[left=1cm,right=1cm,top=1cm,bottom=2cm]{geometry}

\begin{document}

Что все, повидимому, стараются найти указанные в физике причины и помимо этих причин мы не могли бы указать ни одной, - это ясно уже из того, что было сказано раньше.
\footnotemark[1]
Но ставят вопрос о них нечетко. И в известном смысле всо они раньше указаны, а с другой стороны — отнюдь нет.
\footnotemark[2]
Словно лепечущим языком говорит обо всем первая философия1 , будучи в молодых, годах и при начале (своего существования). Ведь и Эмпедокл говорит, что кость существует через соотношение,
\footnotemark[3]
а это означает «суть бытия» и сущность вещи. Но подобным же образом такое «соотношение» (частой) должно быть и у мяса и у всякой другой вещи пли уж — ни в одном случае (без исключения). По этой причине следовательно будет существовать и мясо, и кость, и всякий другой предмет, а не через материю, которую он указывает, — через огонь, землю, воду и воздух. Но с этим он необходимо бы согласился, если бы (так) стал говорить кто другой, сам же он этого отчетливо но сказал.
\footnotemark[4]

Выяснение такого рода вопросов произведено и раньше.
\footnotemark[5]
А все, что по этим же вопросам может вызвать затруднения, мы пройдем снова. Этим путем мы может быть найдем какие-нибудь благоприятные указания для позднейших затруднений.

\end{document}

