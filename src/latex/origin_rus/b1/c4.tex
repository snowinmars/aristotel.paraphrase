\documentclass{article}

\usepackage[T2A]{fontenc}
\usepackage[utf8]{inputenc}
\usepackage[russian]{babel}

\linespread{1.1}
\setlength{\parskip}{1em}
\usepackage[left=1cm,right=1cm,top=1cm,bottom=2cm]{geometry}

\begin{document}

Можно предположить, что Гесиод первый стал искать нечто в этом роде, а также --- если еще кто принял для вещей любовь или страсть в качестве начала, --- как, например, (поступил) и Парменид: ведь и он, устанавливая возникновение вселенной, замечает: «Всех богов первее Эрот был ею замышлен». А по словам Гесиода:
\footnotemark[1]
«В самую первую очередь хаос возник, а затем уж Гея (земля) с широкою грудью»... «также --- Эрот, что меж всех бессмертных богов выдается»1 ; ибо* должна быть среди вещей некоторая причина, которая будет приводить в движение и соединять вещи. Как этих людей распределить [в отношении того], кто (из них высказался по этому вопросу) первый, (об этом) да будет позволено иметь суждение позже; а так как в природе явным образом были (на лицо) и вещи, противоположные хорошим, и не только --- стройность и красота, но также --- нестроение и уродство, причем дурного было больше, чем хорошего, и безобразного --- больше, чем прекрасного, поэтому другой мыслитель ввел дружбу и вражду, выставляя каждую из них как источник [каждого] (соответственного) ряда свойств. В самом доле, если последовать за Эмпедоклом и взять его (слова) по смыслу, а но по тому, что он лепечет в своих речах, можно будет'найти, что дружба есть причина хорошего, а вражда --- причина дурного. И потому, если сказать, что в известном смысле Эмпедокл признает --- и (притом) первый признает --- зло и добро за начала, то это, пожалуй, будет сказано хорошо, поскольку причиною всех благ является (у него) само благо, а причиною зол --- зло.

Перечисленные философы, как мы отмечаем, до сих пор очевидным образом привлекли две причины из тех, которые мы различили в книгах о природе
\footnotemark[2]
 --- материю и источник движения, при этом нечетко и без всякой ясности, но как это делают в битвах люди неискусные; ведь и те, оборачиваясь во все стороны, наносят иногда прекрасные удары, но не потому, что знают; и точно так же указанные философы не производят впечатление людей, знающих, что они говорят: они явным образом совсем почти не пользуются своими началами или в (очень) малой мере. Анаксагор использует ум как машину для создания мира, в когда у него явится затруднение, в силу какой причины (то или другое) имеет необходимое бытпе, тогда он его привлекает, во всех же остальных случаях он все, что угодно, выставляет причиною происходящих вещей, но только не ум.
\footnotemark[3]
А Эмпедокл обращается к причинам больше, нежели Анаксагор, но и он обращается недостаточно и, имея с пими дело, не получает последовательных результатов. По крайней мере у него во многих случаях дружба разделяет, а вражда соединяет. В самом деле, когда целоо под действием вражды распадается •на элементы, тогда огонь собирается вместо и также --- каждый из остальных элементов. Когда же элементы снова под действием дружбы сходятся в единое целое, то из каждого элемента части (его) должны опять рассеяться (в разные стороны).

Эмпедокл, таким образом, в отличие от прежних философов первый ввел разделение (движущей) причины4  --- установил не одно начало движения, а два разных, и притом противоположных. Кроме того, элементы, относимые к разряду материи, он первый указал в числе четырех (он однако же не пользуется ими как четырьмя, а точно их только два: у него (на одной стороне) отдельно --- огонь, а (на другой) противоположные (огню) --- земля, воздух и вода как одно вещество. Это можно было бы вывести, стоя [в своем рассмотрении] на основе его произведений).

Эмпедокл, как мы указываем, установил такие начала и столько их; а Левкипп и его сотоварищ Демокрит признают элементами полное и пустое, называя одно сущим, другое небытием, а именно: полное и твердое --- сущим, а пустое [и разреженное] --- небытием (поэтому они и говорят, что бытие существует отнюдь не более, чем небытие, потому что и тело --- но больше, чем пустота), причиною же вещей является то и другое как материя. И как мыслители, утверждающие единство основной субстанции, все остальное выводят из ее состояний, принимая разреженное и плотное за начала (всех таких) состояний, таким же образом и эти философы считают основные отличия (атомов) причинами всех других свойств. А этих отличий они указывают три: форму, порядок и положение. Ибо бытие по их словам различается лишь «строем», «соприкосновением» и «поворотом»; в том числе «строй», это --- форма, «соприкосновение» --- порядок, «поворот» --- положение; в самом деле, "А" отличается от "Р" формой, "АР" от "РА" порядком, "Ь" от "Р" положением. А вопрос о движении, откуда или в какой форме получится оно у вещей, эти мыслители также, подобно остальным, беспечно оставили в стороне.

Таким образом, относительно двух причин исследование, повидимому, произведено прежними философами в указываемых пределах.

\end{document}

