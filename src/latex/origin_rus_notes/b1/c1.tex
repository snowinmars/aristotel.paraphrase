\documentclass{article}

\usepackage[T2A]{fontenc}
\usepackage[utf8]{inputenc}
\usepackage[russian]{babel}

\linespread{1.1}
\setlength{\parskip}{1em}
\usepackage[left=1cm,right=1cm,top=1cm,bottom=2cm]{geometry}

\begin{document}

\footnotetext[1]{Пол — софист, ученик Горгия. Цитируемое утверждение П. вложено ему в уста в Платоновом «Горгии» (448 С) и находилось в составленном им сочинении (Gorg. 462 В).}

\footnotetext[2]{Я перевожу согласно тексту Ross'a, который подчеркивает (I 117) указание Jacson'a, что под $a$ и $a$ разумеются не больные, а здоровые с определенным темпераментом (флегматическим—холерическим), так что «страдающим такою-то болезнью» ($a$) в примере соответствует лишь в «сильной лихорадке» $a$.}

\footnotetext[3]{Дословно: "случайным образом". По Аристотелевской формулировке врачующий излечивает не человека как такого, но больного, для которого быть человеком — случайное свойство, как и для человека быть больным — случайно.}

\footnotetext[4]{Аристотель дословно говорит о «техниках» (τους  τεχνίτας), имея при этом в виду людей, подобных тем, которые у нас  получили название «ученых специалистов». В соответствии с тем о «мудрости», которая им приписывается, он говорит следующим образом (Elh. Nic. VΙ 7, 1141 а 9—12): «Мудрость в искусствах мы приписываем тем, кто наиболее точно владеет данным искусством, — например, называем мудрым обделывателя камней Фидия и ваятеля Поликлета, разумея  при этом под мудростью не что иное, как то, что она — доброкачественность (αρετή) искусства».}

\footnotetext[5]{У Аристотеля здесь дословно противопоставляется «что» ($\tau o \\ \delta \tau \iota$) и «почему» ($\delta \iota o \tau \iota$) --- факт и причина факта.}

\footnotetext[6]{В скобках [] заключены слова, которых нет в тексте  более ранних рукописей и которые, по-видимому, отсутствовали у Аристотеля.}

\footnotetext[7]{Никомахова этика VI, главы 3—7.}

\end{document}

