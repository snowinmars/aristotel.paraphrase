\documentclass{article}

\usepackage[T2A]{fontenc}
\usepackage[utf8]{inputenc}
\usepackage[russian]{babel}

\linespread{1.1}
\setlength{\parskip}{1em}
\usepackage[left=1cm,right=1cm,top=1cm,bottom=2cm]{geometry}

\begin{document}

\footnotetext[1]{Я перевожу согласно тексту Ross'a, который подчеркивает (I 117) указание Jacson'a, что под $a$ и $a$ разумеются не больные, а здоровые с определенным темпераментом (флегматическим --- холерическим), так что «страдающим такоюто болезнью» ($a$) в примере соответствует лишь в «сильной лихорадке» $a$.}

\footnotetext[2]{Дословно: "случайным образом". По Аристотелевской формулировке врачующий излечивает не человека как такого, но больного, для которого быть человеком --- случайное свойство, как и для человека быть больным --- случайно.}

\footnotetext[3]{Больной необходимо является человеком, но человек бывает больным лишь акцидентально, привходящим образом, ввиду чего лечение данного больного оказывается сущностным, а в его лице человека вообще --- акцидентальпым}

\footnotetext[4]{Под всем остальным, относящимся к тому же роду, Аристотель подразумевает рассудительность, мудрость и ум (см. "Никомахова этика" 1139 b 14 --- 1142 а 30).}

\footnotetext[5]{У Аристотеля здесь дословно противопоставляется «что» ($\tau o \\ \delta \tau \iota$) и «почему» ($\delta \iota o \tau \iota$) --- факт и причина факта.}

\footnotetext[6]{Искусство отличается от науки тем, что оно направлено не на изучение сущего как такового, а на создание вещей. Искусство появляется тогда, когда на сходные предметы вырабатывается единый общий взгляд, объединяющий многие эмпирические представления, благодаря которым приобретается навык, или опыт, возникающий из часто повторяющихся воспоминаний об одном и том же; поскольку же содержание этих воспоминаний составляют восприятия единичных предметов, то и сам опыт имеет дело только с единичным. Следует, однако, иметь в виду, что Аристотель не всегда последовательно проводит различие между «наукой» и «искусством»}

\footnotetext[7]{Под всем остальным, относящимся к тому же роду, Аристотель подразумевает рассудительность, мудрость и ум (см. "Никомахова этика" 1139 b 14 --- 1142 а 30).}

\end{document}

