\documentclass{article}

\usepackage[T2A]{fontenc}
\usepackage[utf8]{inputenc}
\usepackage[russian]{babel}

\linespread{1.1}
\setlength{\parskip}{1em}
\usepackage[left=1cm,right=1cm,top=1cm,bottom=2cm]{geometry}

\begin{document}


\footnotetext[1]{Здесь Аристотель, по-видимому, имеет в виду Анаксимандра, Беспредельное которого, образуя туманную неопределимую ближе массу, являлось первоисточником для всех элементов, принимавшихся в других теориях, и не сводилось ни к одному из таких элементов в отдельности.}


\footnotetext[2]{«То, ради чего», — формальное обозначение для цели, блага (см. главу 3, пр. 1); в нашем случае — в дословном исходном значении.}


\footnotetext[3]{Т.е. считают единое или сущее благом (имеются в виду Платон и его школа), как разбиравшиеся перед тем философы считали благом дружбу или ум. Бониц не прав, указывая (Comm. 98), что слова «приписывают такую (Бон. подчеркивает — «подобного рода») природу» имеют в виду непосредственно предшествующее — «что от них исходят движения».}


\footnotetext[4]{Я перевожу по тексту Ross'a, который, следуя Bywater'y, вместо τούτων пишет τοιούτον.}


\end{document}

