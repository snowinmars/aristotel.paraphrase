\documentclass{article}

\usepackage[T2A]{fontenc}
\usepackage[utf8]{inputenc}
\usepackage[russian]{babel}

\linespread{1.1}
\setlength{\parskip}{1em}
\usepackage[left=1cm,right=1cm,top=1cm,bottom=2cm]{geometry}

\begin{document}


\footnotetext[20]{ Аристотель часто заменяет положительную формулировку понятия неопределенным обозначением его как ответа на тот или иной вопрос. Так, например, --- «то, откуда начало движения» ($todo$), --- источник движения; «то, ради чего» ($todo$), --- цель. Подобный же характер носит его знаменитая формула для обозначения логической сущности: «что есть (или --- что представляет собою, дословно --- чем было) бытие (для такой-то вещи) ($todo$)» --- сущность бытия вещи. Я перевожу эту формулу техническим выражением «суть бытия».}


\footnotetext[21]{ Как отмечает Ross, --- «взятое в конечном счете основание, почему (вещь такова, как она есть)» и «то основное, благодаря чему (вещь именно такова)», весьма неуклюжим образом указывают па одно и то же («формальное») начало}


\footnotetext[1]{ Т. е. ближайшая причина.}


\footnotetext[2]{ Суть бытия (букв, «что именно есть ставшее») --- то, чем является вещь согласно своему определению или что остаетсв ней по отвлечении ее от материи (см. 1017 b 21 --- 22). }


\footnotetext[3]{ Hypokeimenon --- буквально «лежащее в основе» («под-лежащее»). В логике --- субъект. См. «Физика» II 3, 194 Ь 16 --- 195 b 30.}


\footnotetext[4]{ Здесь естество употребляется в значении элемента, стихии (см. 1014 b 27 --- 35).}


\footnotetext[5]{ Букв, «мусическим». К этому термину, сближавшемуся по смыслу со словом «образованный», Аристотель прибегает для обозначения категории качества; в данном случае он хочет сказать, что Сократ не вообще становится, не возникает, а становится другим, т. е. претерпевает качественное изменение.}


\footnotetext[6]{ Океан и Тефия --- родители Океапид (морских божеств). Дословно - "считали отцами".}


\footnotetext[7]{ О Гиппоне, жившем во времена Псрпкла, Александр Афродисийский (Сотга. 428, 21 --- 23) писал, что он прозван «безбожником», поскольку доказывал, что нет ничего помимо чувственно воспринимаемых вещей. Мир, согласно Гпппону, возник в результате преодоления воды образовавшимся из нее огнем.}


\footnotetext[8]{ Диоген из Аполлонии на Крите, выступивший ок. 430 г. до н. э. в поддержку учения Апакспмепа, стремился объяснить физические и психические явления разрежением (нагреванием) и сгущением (охлаждением) воздуха как песотворонпого и беспредельного начала, которое образует и упорядочивает бесконечно сменяющие друг друга миры.}


\footnotetext[9]{ Т. е. что не воздух возникает из воды, а вода возникает из воздуха вследствие его охлаждения и сгущения. }


\footnotetext[10]{ Простые тела --- четыре элемента: земля, вода, воздух и огонь}


\footnotetext[11]{ Гиппас из Метапонта (V в. до н. э.) отождествлял пифагорейский центральный огонь с гераклитовским первовеществом}


\footnotetext[12]{ К воде, воздуху и огню.}


\footnotetext[13]{ По учению Эмпедокла, элементы не могут превращаться друг в друга или, соединиптись, образовывать новый элемент; они способны лишь в той или иной пропорции смешиваться друг с другом и рассеиваться под воздействием сил «дружбы» («любви») н «вражды».}


\footnotetext[14]{ Однородные частицы --- то, что Анаксагор называл семенами вещей, разумея под ними лежащие в основе всего бесчисленные невозникшие, непреходящие и неизменные тельца с однородной структурой, соответствующей определенному качеству.}


\footnotetext[15]{ Причину движения.}


\footnotetext[16]{ «Теплое» и «холодное».}


\footnotetext[17]{ Гермотим из Клазомеп слыл чудотворцем, душа которого пособии па длительное время покидать свое тело и вновь возращаться в него.}


\footnotetext[18]{ Т.е. в книгах "Физики".}


\footnotetext[19]{ Буквально - "природа", в данном случае --- элемент или стихия. }

\end{document}

