\documentclass{article}

\usepackage[T2A]{fontenc}
\usepackage[utf8]{inputenc}
\usepackage[russian]{babel}

\linespread{1.1}
\setlength{\parskip}{1em}
\usepackage[left=1cm,right=1cm,top=1cm,bottom=2cm]{geometry}

\begin{document}


\footnotetext[1]{см. 981 Ь 28—«так называемая мудрость».}


\footnotetext[2]{А именно — потенциально, в возможности (Ross).}


\footnotetext[3]{ Или: «которая охватывается этим знанием». Дословно: «все, что подлежит (подведомственно) <этому знанию >».}


\footnotetext[4]{Противоречие между утверждением Аристотеля, что  первые начала наиболее трудны для познания, и утверждением, что они наиболее познаваемы, снимается тем, что эти начала, согласно обычному аристотелевскому различению, представляют наибольшие затруднения для  нас, с нашей обычной точки зрения, так как для нас всего ближе непосредственная очевидность чувственного восприятия, но они лучше всего постигаются нашею мыслью, так как все более конкретное наше  знание покоится на достоверном познании этих первопринципов нашим  разумом.}


\footnotetext[5]{Дословно: «науке в наибольшей степени»; «науке по преимуществу» (Роз. и Перв.). Аристотель хочет сказать: науке, которая является таковой в наибольшей степени, т.е. в наибольшей мере имеет характер науки.}


\footnotetext[6]{В этом месте термин «подлежащее» (ύποχεί-μενον), в первую очередь означающий у Аристотеля «то, что лежит в основе», «субстрат», может быть принят в своем дословном значении — то, что лежит внизу чего-нибудь, т.е. то, что зависит от чего-нибудь или подчинено чему-нибудь.}


\footnotetext[7]{Выражение «то, ради чего» Аристотель употребляет как технический термин для обозначения цели.}


\footnotetext[8]{παθήματα — «испытываемые состояния». В первую очередь здесь, вероятно, имеются в виду фазы луны и затмения солнца.}


\footnotetext[9]{Дословно: «всего» (τοο παντός), но не в смысле всех вещей, а в смысле всеобъемлющего целого.}

\footnotetext[10]{Т.е. находится в зависимости от внешней обстановки.}

\footnotetext[11]{Симонид — греческий поэт, лирик, родом с острова Ксоса (2-я половина VI века — начало V века до нашей эры).}

\footnotetext[12]{Слова «кто еще не рассмотрел причину» в тексте рукописей стоят после слов: «про загад. самодвиж. игрушки». Ross в своем издании переносит их на то место, где они стоят в нашем переводе, ссылаясь на авторитет Jager'a и Bonitz'a (Bon., правда, сам этой перестановки сделать не решился, — Comm. 57).}

\end{document}

