\documentclass{article}

\usepackage[T2A]{fontenc}
\usepackage[utf8]{inputenc}
\usepackage[russian]{babel}

\linespread{1.1}
\setlength{\parskip}{1em}
\usepackage[left=1cm,right=1cm,top=1cm,bottom=2cm]{geometry}

\begin{document}


\footnotetext[1]{ Эйдосами (или «идеями») Платон называл вечные и неизменные умопостигаемые прообразы вещей (их роды и виды), запредельные по отношению к преходящим и изменяющимся чувственно воспринимаемым предметам, которые существуют через «причастность» этим своим прообразам. В настоящем издании eidos переводится как «эйдос», когда речь идет об «идеях» Платона, как «форма», когда говорится о материи и форме, и как «вид», когда eidos рассматривается наряду с родом и индивидом.}


\footnotetext[2]{ Одноименные с эйдосами вещи --- это чувственно воспринимаемые предметы, разделяющие название и суть бытия объемлющих их родов и видов.}


\footnotetext[3]{ В учении Платопа большое и малое --- неоформленное пространство, выступающее в качестве материальной «сопричины» существования телесного мира.}


\footnotetext[4]{ Диалектика, согласно Платону, --- это высшая наука о сущем, искусство побуждать к исследованию посредством указания противоречий в обычных мнениях о вещах и в то же время метод постижения истины. Этот метод предполагает, с одной стороны, обнаружение в многообразном общего и единого, восхождение от первоначальных предположений к их умопостигаемым основаниям, вплоть до наивысшего --- блага, а с другой --- нисхождение к исходным предположениям через деление родов на виды --- вплоть до «неделимого».}


\footnotetext[5]{ Речь, по-видимому, идет о непроизводных математических числах.}


\footnotetext[6]{ По представлениям Платона и Аристотеля, мужское начало --- формообразующий принцип, а женское --- материальный}

\end{document}

