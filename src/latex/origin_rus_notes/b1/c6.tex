\documentclass{article}

\usepackage[T2A]{fontenc}
\usepackage[utf8]{inputenc}
\usepackage[russian]{babel}

\linespread{1.1}
\setlength{\parskip}{1em}
\usepackage[left=1cm,right=1cm,top=1cm,bottom=2cm]{geometry}

\begin{document}


\footnotetext[1]{В отличие от Воn. я связываю вместе не των όντων ιδέας, а τά τοιαύτα των όντων}


\footnotetext[2]{Ross (I 161) высказывается против того, чтобы здесь подразумевать «существуют», и ставит παρά ταύτα в зависимость от λέγεσ$αι, причем παρά здесь толкует аналогично с его значением в слове παρώνυμος: «а чувственные вещи обозначаются в зависимости от них и сообразно с ними». Но тогда разница в смысле между παρά ταύτα и $ατά ταύτα оказывается слишком незначительной.}


\footnotetext[3]{Следуя Gillesple и Ross'у (I 161—2), я исключаю τοις εΐδεσιν и ставлю των συνωνύμων (под «одноименными сущностями» разумеются однородные чувственным вещам и по имени и по сущности идеи) в зависимость не от τά πολλά, а от χατά μέθεξιν.}


\footnotetext[4]{Слова «изменивши имя» Christ вслед за одною из рукописей (А^b) выпускает из текста, как ненужное повторение только что сказанного, попавшее в текст впоследствии.}


\footnotetext[5]{Если не считать вместе с Christ'ом «числа», а вместе с Gillespie и Целлером «идеи» позднейшей вставкой, то надо видеть в числах пояснение к «идеям». Это дает вполне приемлемый смысл, ибо «единое», с одной стороны, «большое и малое» — с другой, выступили у Платона как основные начала в тот период, когда идеи были у него сведены к числам.}


\footnotetext[6]{Или: «и точно так же он разделял с ними мнение, что...».}


\footnotetext[7]{Некоторые ученые (Тренделенбург, Швеглер) под «первыми» числами, которые с платоновской точки зрения не могут быть выводимы из первичной «двойки», разумели числа идеальные, а Целлер считает, что слова «за исключением первых» должны быть удалены из текста, как позднейшая вставка. Но, как справедливо указывает Бониц (Comm. 94—95), речь идет здесь не об идеальных числах как таких (для конструкции которых и введена специально «двоица»), а о непроизводных математических числах: по Платону «двоица», есть принцип всякого повторения, и, следовательно, она бесполезна там, где число не образовано с помощью повторения (т.е. — какой-либо комбинации четных или нечетных чисел).}


\footnotetext[8]{Или, может быть, «то, что Платон называет «идеей».}


\footnotetext[9]{Или: «и однако мы имеем здесь подобное же положение, как и при тех началах».}


\footnotetext[10]{Т.е. с единым (причину добра) и с материальною двойственностью (причину зла). Дословно у Аристотеля: «причину добра и зла он указал в элементах — в каждом из двух одну из двух» (т.е. в одном — одну, в другом — другую).}


\footnotetext[11]{Вместо «как, по нашим словам, искали ее...» можно было бы сказать: «мы напоминаем, что таким же образом ее искали» и т.д.}


\end{document}

