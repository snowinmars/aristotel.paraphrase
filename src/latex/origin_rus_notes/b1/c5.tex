\documentclass{article}

\usepackage[T2A]{fontenc}
\usepackage[utf8]{inputenc}
\usepackage[russian]{babel}

\linespread{1.1}
\setlength{\parskip}{1em}
\usepackage[left=1cm,right=1cm,top=1cm,bottom=2cm]{geometry}

\begin{document}


\footnotetext[1]{ В учении пифагорейцев — небесное тело, расположенное между землей и центральным космическим огнем}


\footnotetext[2]{ См. «О небе» II 13, 293 а 20 — b 3. Возможно, речь так же идёт об утраченной работе "Взгляды пифагорейцев". }


\footnotetext[3]{ В последнем случае противопоставляются квадрат и разносторонний прямоугольник}


\footnotetext[4]{ Алкмеон из Кротона (нач. V в. до и. э.) — врач, прославившийся своими анатомическими исследованиями, изучением органов чувств и сделанным им открытием, что мозг — центральный орган разумной деятельности. Алкмеоп, в частности, утверждал, что па человеческий организм воздействуют противоположные силы л что задача «рача состоит поэтому в том, чтобы поддерживать составные элементы тела в равновесии.}


\footnotetext[5]{ Physiologoi — «фисиологи», мыслители, исследовавшие природу}


\footnotetext[6]{ «Одно и то же есть мысль и бытие» (Парменид. О природе, V 1).}


\footnotetext[7]{ См. «Физика» 1 2, 184 b 15-186 а 3.}


\footnotetext[8]{ Т. е. пифагорейцев (основанная Пифагором школа находилась в Кротоне, городе на юге Италии). }


\footnotetext[9]{ Речь идет, видимо, о двух противоположных началах («предел» и «беспредельное») или о числе как материи для сущею, с одной стороны, и как выражении его состояний и свойств — с другой.}


\footnotetext[10]{ Т. е. того, что определяется как беспредельное и единое.}


\footnotetext[11]{ То ti einai. Букв, «что именно есть [данная вещь]».}


\footnotetext[12]{ У пифагорейцев нередко получалось так, что один и тот же предмет обозначался несколькими числами и, наоборот, одно и то же число обозначало несколько предметов (число 4, например, служило для обозначения дружбы, справедливости, тела)}


\footnotetext[13]{ В доплатоповские времена.}

\end{document}

