\documentclass{article}

\usepackage[T2A]{fontenc}
\usepackage[utf8]{inputenc}
\usepackage[russian]{babel}

\linespread{1.1}
\setlength{\parskip}{1em}
\usepackage[left=1cm,right=1cm,top=1cm,bottom=2cm]{geometry}

\begin{document}

\footnotetext[1]{См. «Физика» II 3,194Ь 16 --- 195Ь 30.}

\footnotetext[2]{Причины правильно назывались, но неверно толковались.}

\footnotetext[3]{Через соотношение входящих в ее состав элементов.}

\footnotetext[4]{Эмпедокл рассматривает четыре элемента как субстанции вещей, и в качестве таковых их можно было считать материальными причинами; но, говоря об их соотношении как о сущности вещи, он, вряд ли отдавая себе в этом ясный отчет, по сути дела говорит о формальной причине.}

\footnotetext[5]{Вопросы, касающиеся первых начал, рассматривались в «Физике» \(кп. II, гл. 3 и 7\).}

\end{document}

