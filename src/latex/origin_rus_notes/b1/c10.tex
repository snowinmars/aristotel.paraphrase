\documentclass{article}

\usepackage[T2A]{fontenc}
\usepackage[utf8]{inputenc}
\usepackage[russian]{babel}

\linespread{1.1}
\setlength{\parskip}{1em}
\usepackage[left=1cm,right=1cm,top=1cm,bottom=2cm]{geometry}

\begin{document}

\footnotetext[1]{«Первая философия» здесь в смысле «основная философия» (т.е. наука о первых началах, метафизика), как Аристотель неоднократно употребляет этот термин (Met. Ε 1, 1026 а 16, Phys. А 9, 192 а 36 и др.).}

\footnotetext[2]{Ввиду накопления однозначных выражений (дословно: «будучи вначале и впервые в молодых годах»), Ross рекомендует читать: «будучи в молодых годах и при начале» (άτε νέα τε χα! κατ'άργάς ούσα), а «и впервые» (και τό πρώτον) считает позднейшим пояснением (глоссой) для «при начале» (κα! κατ'αοχάς).}

\footnotetext[3]{Аристотель говорит — через «логос», имея в виду указываемое Эмпедоклом соотношение отдельных материальных элементов в составе кости. При многочисленности значений термина «логос», его нельзя здесь передавать через субъективное «понятие», хотя, с другой стороны, «соотношение» (у Ross'a — ratio) плохо вяжется с моментом субстанциальности, который характерен для привлекаемой Аристотелем «сущности» и «сути бытия».}

\footnotetext[4]{Дословно: «или уж — ни у какой». Здесь я читаю по тексту Christ'a, который ближе к чтению рукописей, так как не вижу решающих оснований для предложенной Воn. и принятой Ross'ом конъектуры, по которой вместо $α$ σαρκός $α$ των άλλων (εκάστου) είναι τον λόγον ή μηδενός, надлежит читать $α$ σάρκας $α$ των άλλων εκαστον είναι τον λόγον, ή μηδέ εν. Непонятно, почему при первом чтении требуется (Ross I, 213) мысленно добавлять ούσίαν или φύσιν, а нельзя понимать указанное место (как это и выражено в переводе) так: «указанное соотношение должно быть и у мяса и у каждой из других вещей» и т.д.}

\footnotetext[5]{Под этим «прежним выяснением» надо иметь в виду не «все, что о высших началах вещей исследовали или смутно предугадывали древние философа», как думает Воn. (Comm. 127), но то, что раньше было уже сказано о первых началах самим Аристотелем (см. в первую очередь «Физику», книга II, главы 3, 7).}

\footnotetext[6]{Alex. поясняет: «Затруднения и их решения в отношении начал становятся исходным пунктом для решения последующих затруднений» (Comm. 100, 3—4).}

\end{document}

