\documentclass{article}

\usepackage[T2A]{fontenc}
\usepackage[utf8]{inputenc}
\usepackage[russian]{babel}

\linespread{1.1}
\setlength{\parskip}{1em}
\usepackage[left=1cm,right=1cm,top=1cm,bottom=2cm]{geometry}

\begin{document}


\footnotetext[1]{Дословно: «имеющею причину»}


\footnotetext[2]{Дословно: «рассматривая (изучая) природу в отношении всех вещей». Роз. и Первов переводят: «... рассуждая о природе всех вещей...»}


\footnotetext[3]{Заключенные в скобки слова Christ выбрасывает из текста, считая их позднейшей вставкой, перебивающей ход мысли.}


\footnotetext[4]{Точнее: «с наиболее мелкими частями и наиболее тонкое».}


\footnotetext[5]{Дословно: «первою из тел».}


\footnotetext[6]{Слово «популярный», по-видимому, не только дает дословный перевод, но и вполне точно передает тот оттенок, который здесь имеет у Аристотеля слово δημοτικός.}

 
\footnotetext[7]{Аристотель имеет в виду De coelo III, так что указание на «сочинения о природе» не всегда относится специально к «Физике».}


\footnotetext[8]{При более точном (почти дословном) переводе — неправильная конструкция, как и в греческом тексте. «А про Анаксагора если бы кто счел, что он принимает два элемента, то он оказался бы в наибольшем соответствии с (верным) ходом мысли, которого сам он, правда, не расчленил, но с необходимостью последовал бы за теми, кто стал бы указывать ему путь (склонять его за собой)».}


\footnotetext[9]{Определенность «по существу» означает у Аристотеля определенность по первой из категорий (каковой у него является категория сущности), в отличие от определенности по качеству или количеству.}


\footnotetext[10]{Аристотель здесь уже говорит языком своей философии, по которой принципом всякой определенности является та или иная форма, в противоположность неопределенности материи.}


\footnotetext[11]{«Мы признаем» сказано от имени Платоновской школы, к которой Аристотель — в качестве одного из учеников Платона — часто себя причисляет (даже тогда, когда он специально занимается критикой платоновских учений, ср. А 9, 991 Ь 7).}


\footnotetext[12]{Дословно: «К тому, что более представляется», по-видимому наиболее просто дополнить «правильным», «приемлемым». Стоящее кроме того в рукописях νυν (теперь) Bonitz и Christ опускают, так как оно не оговаривается у Alex., но слово это, по-видимому, вполне может быть сохранено, поскольку оно специально указывает на современные (может быть, даже на аристотелевские) взгляды.}


\footnotetext[13]{Т.е. лишь относительно сущности, подверженной возникновению, уничтожению и движению.}


\footnotetext[14]{Для передачи общего хода развития мысли фразу можно сформулировать так: «Прежде всего так называемые пифагорейцы пользуются более необычными началами и элементами...» (этому «прежде всего», по-видимому, соответствует затем начало главы 9).}


\footnotetext[15]{Дословно: «физиологи» (см. примечание 13 к 5-й гл. I книги).}


\footnotetext[16]{Я перевожу «астрономии», чтобы избежать связавшегося у нас с термином «астрология» отрицательного оттенка; у греков употреблялись оба термина — Аристотель говорит почти исключительно «астрология», Платон в «Государстве» «астрономия» (Politeia VII, главы 10—11).}


\footnotetext[17]{Дословно: «они нисколько не больше относят свои слова к математическим телам, чем к чувственно-воспринимаемым». Иными словами: если из предела и беспредельного еще можно с натяжкою получить протяженность, то как же из них вывести тяжесть? А между тем эти начала должны у них все объяснять одинаково, — математические свойства тел не больше, чем физические. — Таким образом, Швеглер и Бониц правы, признавая ненужной перестановку Казавбона, который хотел читать: «они не говорят о чувственных вещах ничего больше, чем о математических».}


\footnotetext[18]{Начинающееся отсюда крайне трудное место (990 а 18—29), истолковать которое в свое время отказался Бониц, теперь можно считать в значительной мере разъясненным совместными усилиями ученых. Аристотель хочет показать то своеобразное затруднение, которое получается для пифагорейцев, поскольку они отождествляют вещи с числами, а не считают числа образцами для вещей, как это делает Платон. Вселенная есть число или некоторая совокупность чисел; отдельные части вселенной — отдельные числа, которые реализованы в этих частях. Поэтому, считая сущностью различных вещей те или другие числа, пифагорейцы помещали эти вещи в те части вселенной, которые являли собою реализацию данных чисел (например, «мнение» помещалось на землю, потому что число мнения есть два, и в то же время два есть число земли, вернее — число группы, состоящей из земли и центрального огня). Таким образом, естественно встает вопрос, имеет ли (если даже держаться пифагорейской концепции) значение для той или другой вещи (мнения, несправедливости и т.д.) осуществление составляющего эту вещь числа в каком-нибудь определенном месте вселенной (как это странным образом выходило у пифагорейцев, совмещавших предметы совершенно разнородные), или же истинною природой всякой вещи они должны бы были считать логическое существо ее числа, независимо от космической роли этого числа (ведь и пифагорейцы учили, что вещи подражают числам, и, следовательно, разные вещи могли выявлять собою одни и те же числа, но с разных точек зрения, — или непосредственно имея эти числа своею сущностью, или представляя собою конкретное космическое их отображение).}


\footnotetext[19]{Ross (I 185—186), давая здесь несколько иное объяснение, предлагает толковать «составных <материальных> величин», имея при этом в виду элементарные тела (стихии), «составляемые» из того или другого числа и находящиеся в разных местах мира.}


\footnotetext[20]{Так — по конъектуре Целлера, предложившего здесь читать διό вместо oti τό. Однако же, и в случае оставления засвидетельствованного рукописями текста (Christ и Ross) получается, может быть, достаточно удовлетворительный смысл: «... вследствие того, что указанные явления и т.д.». В таком случае уже — обратный порядок: в определенном месте появляется определенное количество небесных тел, потому что оно отмечено определенным числом и наличием соответствующих вещей (или явлений).}


\footnotetext[21]{Проще и достаточно точно по смыслу будет перевести: «то надо ли признать, что здесь мы имеем это же самое находящееся на небе число, которое доставляет каждое данное явление и т.д.».}

\end{document}

