\documentclass{article}

\usepackage[T2A]{fontenc}
\usepackage[utf8]{inputenc}
\usepackage[russian]{babel}

\linespread{1.1}
\setlength{\parskip}{1em}
\usepackage[left=1cm,right=1cm,top=1cm,bottom=2cm]{geometry}

\begin{document}


\footnotetext[1]{ По свидетельству Аэция (IV 5), Ксенофан считал, что «из земли все [возникло] и в землю все обратится в конце концов».}


\footnotetext[2]{ См. «О небе» III 7, 305 а 33 --- 306 Ь 2.}


\footnotetext[3]{ Определенным нечто (tode ti) Аристотель называет нечто конкретное, на что можно указать как на «вот это». Т. е. определенность по первой из категорий (по категории сущности), в отличие от определенности по качеству или количеству. }


\footnotetext[4]{ Здесь и в некоторых других случаях далее Аристотель говорит от имени платоновской школы, причисляя к ней себя как ученика ее основателя.}


\footnotetext[5]{ В отношении сущности, которая возникает, уничтожается и движется.}

\end{document}

