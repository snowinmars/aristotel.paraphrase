\documentclass{article}

\usepackage[T2A]{fontenc}
\usepackage[utf8]{inputenc}
\usepackage[russian]{babel}

\linespread{1.1}
\setlength{\parskip}{1em}
\usepackage[left=1cm,right=1cm,top=1cm,bottom=2cm]{geometry}

\begin{document}


\footnotetext[1]{Дословно: «для которых ища причин, от них пришли к идеям».}


\footnotetext[2]{Я сохраняю текст, установленный Christ'ом, ибо текст Ross'a (κα&'έχαστον γάρ δμώνυ[Λ0ν τι έστι και παρά τάς ουσίας, των τε άλλων εστίν εν έπι πολλών) — «в отношении к каждой вещи (или, может быть, к каждому роду вещей) есть нечто одноименное и помимо сущностей, и для всего остального есть единое во многом» — повторяет второй фразой первую, если не толковать «и помимо сущностей» весьма натянутым образом («есть нечто одноименное и существующее помимо <здешних> сущностей»), на что, однако, у Ross'a как будто есть намек («an entity existing apart from the substances»), Ross I 191).}


\footnotetext[3]{Как поясняет Alex. (Comm. 58, 15 —16), «слова «мы доказываем» обнаруживают, что, передавая суждения Платона, он говорит, как против близкого (родного) мнения».}


\footnotetext[4]{Alex. (ib. 58, 27—29) приводит такие примеры «аргументов, из которых не получается силлогизма»: «если есть какая-нибудь истина, то надо думать, что существуют идеи, ибо из здешних вещей ничто не истинно; и если есть память, то есть идеи, ибо намять имеет своим предметом то, что пребывает».}


\footnotetext[5]{Отдельные аргументы в пользу идей, очевидно, постепенно получили в школе Платона сокращенные условные обозначения, и Аристотель, обращаясь в своем «докладе» к товарищам по школе — именно такой характер, по-видимому, первоначально носила составившая затем первую книгу всей «Метафизики» работа, — указывает те или другие из этих аргументов посредством их общеизвестных «школьных» наименований. В частности, «доказательства от наук» исходили от утверждения, что объект науки должен быть устойчивым и носить общий характер, между тем у чувственных предметов этих свойств нет; таким образом, выдвигается требование особого, отдельного от чувственных вещей предмета. Аристотель подчеркивает, что при этих условиях все рассматриваемое науками общее должно было превратиться в идеальные предметы; между тем в ряде случаев общее остается общим, не получая гипостазирования (так — в отношении всех объектов техники фактическую реальность — и с точки зрения Платоновской школы — можно приписывать только индивидуальному).}


\footnotetext[6]{«Единое, относящееся ко многому», приводит к неприемлемым для Платоновской школы выводам в том, например, случае, если одно и то же определение отрицается в отношении ряда предметов. Так, например, «не человек» или «не образованный» есть общее обозначение, высказываемое относительно множества объектов (лошади, собаки и т.д.), но такой идеи школа не устанавливает.}


\footnotetext[7]{Указывая на «наличие объекта у мысли по уничтожении вещи», Платон требовал признания идей, поскольку, независимо от уничтожения ряда преходящих объектов, мысль человека продолжает иметь перед собою некоторое содержание, общее всему — пусть даже всецело уничтожившемуся — ряду. Но если объективировать такое содержание, то на этом же основании приходится требовать также объективации содержания всякого индивидуального предмета, поскольку это содержание сохраняется в той или иной мысли — уже по уничтожении данного предмета, и, следовательно, будет столько идей, сколько было индивидуальных вещей.}


\footnotetext[8]{Ross (I, 194), со ссылкой на Джэксона, отмечает, что в предшествующем анализе Аристотелем были указаны противоречащие платоновскому учению выводы, сделанные Аристотелем из платоновских аргументов в пользу идей; теперь Аристотель констатирует затруднения, связанные с более разработанными доказательствами, — затруднения, которые сам Платон сознавал и которые, не создавая внутренних противоречий, все же побуждали школу к нежелательным для нее выводам.}


\footnotetext[9]{Ross приписывает здесь Аристотелю мысль, что с точки зрения Платоновской школы нельзя выделить соотнесенных предметов в одну самостоятельную группу («мы — имеются в виду платоновцы — не предполагаем, что все вещи, которые случайно оказываются равными другим вещам, образуют отдельный класс среди вещей (in rerum natura)», I 194). На самом деле, по-видимому, Аристотель говорит не о том, что школа не считала возможным создавать из соотнесенных предметов самостоятельную группу, а о том, что по ее основному взгляду отношения не могут образовать такую группу, — группу, которая должна бы была мыслиться отдельно от соотнесенных предметов.}


\footnotetext[10]{Аргумент относительно «третьего человека», выдвигавшийся еще непосредственно против Платона и учтенный им в диалоге «Парменид», требовал принимать идею, общую ряду однородных индивидуальных вещей, с одной стороны, и идее ряда этих вещей — с другой. Таким образом, оказывалось необходимым продолжить в бесконечность тот путь, который первоначально объединил в одной идее ряд индивидуальных вещей.}


\footnotetext[11]{Если некоторая идея (например, вечности) не существует самостоятельно, а есть только свойство другого — самостоятельного — бытия («высказывается о подлежащем»), тогда в этой области все вещи надо было бы в первую очередь ставить в зависимость от того основного бытия, коего наша идея является свойством, и они определялись бы через природу этого бытия, а к нашей идее находились бы только в случайном отношении, т.е. выражаемое ею свойство отнюдь не было бы им с необходимостью присуще, — так, например, если мы в качестве самостоятельного бытия возьмем двойное в себе, тогда вещи, причастные двойному, отнюдь еще не должны быть (а только иногда случайно могут быть) вечными, хотя вечность и есть одно из свойств двойного в себе.}


\footnotetext[12]{Бониц предлагает здесь внести изменение в текст рукописей и переводит это место: «Таким образом, идеи будут идеями сущностей» (ώστ'έσται ουσιών τά εΐδη), считая, что Аристотель уже в этот момент приходит к намеченному им выводу («по логической необходимости... должны существовать только идеи сущностей»). Christ и Ross не принимают в свой текст Вonitz'евской конъектуры, и она не будет нужна, если признать, что в данный момент цель Аристотеля — только подчеркнуть необходимость для идей быть сущностями, поскольку они — предмет приобщения (Bon. исходит из предположения, что идеи — всегда сущности). Общий ход мысли Аристотеля, по-видимому, такой: «Идеи должны быть только идеями сущностей. Это вытекает из того, что вещи к ним причастны. Дело в том, что вещи должны быть причастны к идеям самим по себе (в их самостоятельном бытии), а не поскольку они — свойства другого (не поскольку они «высказываются о подлежащем»), ибо в этом последнем случае определение вещи через идею было бы случайным, вещь основным образом определялась бы через то другое, о чем «сказывается» такая идея, и выражаемое идеею свойство могло бы и не принадлежать вещи, вещь находилась бы к этой идее в случайном отношении. Значит, идеи, поскольку им причастны вещи, должны быть сущностями. А отсюда (это уже следующий шаг) — вещи, причастные идеям (как сущностям), также должны быть сущностями. — И только при этом условии (если те и другие, как сущности, принадлежат к одному роду) между ними может быть действительная связь, в противном случае между ними будет общность только по имени. — Таким образом, та связь, которая первоначальными аргументами устанавливалась между всякого рода множеством и объединяющей его идеей, теперь объявляется совершенно нереальною.»}


\footnotetext[13]{είδος — здесь опять в смысле — группа, разряд (как и в выражении έν ύλης ehsi). У идей и причастных им вещей будет один είδος, поскольку «один и тот же смысл имеет сущность и здесь и там» (см. выше).}


\footnotetext[14]{Имеется в виду сопоставление всякого рода чувственно-воспринимаемых пар, с одной стороны, и математических двоек, которых может быть неопределенно много совершенно одинаковых, с другой (см. выше 987 Ь 16—17).}


\footnotetext[15]{В том смысле, как указано выше, в примечании 13.}


\footnotetext[16]{Т.е. поскольку они выступают не в качестве имманентных начал (как это было у Анаксагора и Евдокса), а в собственном — платоновском — смысле, в качестве начал трансцендентных.}


\footnotetext[17]{Аристотель имеет в виду главу 24 книги V, где указаны различные значения для выражения «быть из чего-нибудь», и констатирует, что ни одно из этих значений не годится, чтобы формулировать отношение между идеями и здешним бытием.}


\footnotetext[18]{Я читаю по указываемой у Ross'а конъектуре Richards'а α 24 ότωουν («с чем угодно»), вместо рукописного и принятого в изданиях ότιοον («... что угодно может и быть и становиться сходным»), так как иначе дальнейшее πρός έ$εΐνο («с него») остается без объекта, к которому оно бы относилось.}


\footnotetext[19]{Или: не сочтем ли мы (невозможным).}


\footnotetext[20]{Т.е. естественно существующие вещи (τά φύσει).}


\footnotetext[21]{Вещи, создаваемые искусством (τά τέχντ,).}


\footnotetext[22]{Коих идеи — числа должны быть причинами.}


\footnotetext[23]{Чисто грамматически здесь конструкция представляется неправильной (δήλον δτι εστίν ένγέ τ», ών είσι λόγοι <τάνταύθα>), и поэтому Walker, как отмечает Ross, предложил читать ου вместо ών. Однако, Ross считает здесь множественное число приемлемым и даже более желательным, указывая, что отношения требуют двух соотносящихся членов, и все место объясняет так: очевидно, что к чему-то одному сводятся (Ross дословно говорит — «какой-нибудь один класс вещей образуют») те элементы (у Ross'a— the things I 200), между которыми здешние вещи представляют собою то или другое отношение.}


\footnotetext[24]{Я читаю по тексту Christ'а εί oh τι-τούτο, ή υλη, имеющему хорошую рукописную базу и по смыслу здесь заслуживающему предпочтения перед текстом Ross'a ει δή τούτο ή ΰλη (6 14—15).}


\footnotetext[25]{Т.е. идеальные числа.}


\footnotetext[26]{Как указывают и Bon. и Ross, здесь по ходу аргументации естественно было бы сказать: «идея будет отношением в числах каких-нибудь других... вещей». Но Ross вместе с тем отмечает, что рукописный текст можно удержать, потому что в данном месте центр тяжести — в словах «каких-нибудь других лежащих в основе вещей», и Аристотель здесь называет идею числом, еще пользуясь пока терминологией Платона (или, может быть, поскольку число будет уже в этом случае выражением отношения между числами).}


\footnotetext[27]{Идеи, если они — числа, всегда должны, как и здешние вещи, представлять числовое отношение тех или других субстратов. При этом в одних случаях они могут выступать как числа (например, число октавы — 2, потому что отношение октавы 2:1), в других — как отношения чисел (например, число мяса или кости — 3 части огня, 2 части земли, см. Met. XIV, 5), но всегда в основе своей будут именно отношениями.}


\footnotetext[28]{Эти последние слова, по-видимому, (ср. Ross I 200) специально означают: и из-за того, что некоторые идеи носят характер чисел, не следует, чтобы в идеальном мире существовали на самом деле какие-нибудь числа.}


\footnotetext[29]{Здесь имеются в виду идеальные числа.}


\footnotetext[30]{Здесь имеются в виду идеальные числа.}


\footnotetext[31]{Как указывает Alex. (Comm. 82, 9 след.), главные из этих нелепостей сводятся к тому, что в таком случае идеальные числа будут только суммами разного количества единиц, и, таким образом, качественно различные вещи будут объясняться лишь количественно разнящимися между собою принципами.}


\footnotetext[32]{Ross (I, 200) рекомендует, следуя Bywater'у, читать вместо αί αύται—αύται, чтобы речь шла не о «тех же самых единицах», а о самих единицах, находящихся в данном числе.}


\footnotetext[33]{Ибо, согласно основному предположению, каждая единица будет индивидуально своеобразна (единицы — разнородны) и будет возникать из основного единого и основной неопределенности («неопределенная двойка», как принцип такой неопределенности).}


\footnotetext[34]{Опять — поскольку единицы разнородны, как они могут связаться в единое целое?}


\footnotetext[35]{Поскольку единицы — тем более числа — признаются качественно различными, нужно при объяснении мира исходить из них, — по образцу натурфилософов, принимавших несколько различных элементов (не возводя их к одной материи). Между тем, Платоновская школа считает первоначальное единое как бы однородным со всеми единицами, которые из него получаются, ибо каждая из этих единиц являет основную природу единого (Аристотель едва ли правильно представляет себе установленную Платоном роль единого: на самом деле единое у Платона — творческий принцип, создающий на основе неопределенности «диады» (двойки) конкретные числа). Отсюда созданные единым единицы — и точно так же идеальные числа — не могут быть самостоятельными сущностями. Очевидно поэтому, что если единое должно быть началом, то у него должно быть несколько значений (в одном случае единое — только общий род для качественно различных единиц, тогда причинами вещей будут эти качественно различные единицы; в другом — единое есть нечто существующее в себе, — тогда отдельные единицы не являются сущностями, и началом будет только это единое).}


\footnotetext[36]{Дается критика той дедукции, которую в Платоновской школе получают основные роды геометрических величин (как говорит Ross, они были бы идеями, если бы идеями не признавались в теперешней стадии развития школы только числа): величины, это — непосредственно следующие за идеями «идеальные» моменты, которые выводятся из «идей» — идеальных чисел 2, 3, 4, с одной стороны, и из видоизменений основного материального принципа (большого и малого) — с другой: линии — из длинного и короткого, плоскости — из широкого и узкого, тела — из глубокого и низкого.}


\footnotetext[37]{Аристотель опять говорит от имени Платоновской школы.}


\footnotetext[38]{μήκη означает величины одного измерения, иногда = линии.}


\footnotetext[39]{В противоположность «глубокому» мы бы сказали «мелкого».}


\footnotetext[40]{При наличии в основе каждого из этих геометрических родов (линии, плоскости, тела) особых начал.}


\footnotetext[41]{В основных геометрических определениях — линии, плоскости, теле.}


\footnotetext[42]{Имеются в виду источники геометрических определений: длинное и короткое и т.д.}

\footnotetext[43]{Ross в результате анализа этого места (I 203—208; ср. также I 189) предлагает другую пунктуацию, которая дает такой смысл: «... с этим родом бытия и боролся Платон, как с геометрическим учением, а началом линии называл (и неоднократно он это <начало> указывал) неделимые линии» ('αλλ' έκάλει αρχήν γραμμής — τώτο δέ πολλάκις έτί$ει — τάς άτόμους γραμμάς, α 20—22).}


\footnotetext[44]{От имени Платоновской школы.}


\footnotetext[45]{Имеется в виду, как явствует из непосредственно следующего, причина целевая. Ross не считает здесь нужным изменять чтение рукописей, указывая, что наша фраза хочет только подчеркнуть общее значение этой причины («to emphasize its importance») для научного познания. Может быть, даже, чтобы определеннее выявить основной смысл текста, достаточно более тесно слить две рядом стоящие фразы: «... что касается имеющей значение для наук (играющей роль для наук) причины, ради которой творит...» — Rolfes предлагает читать: «та причина, которая имеет значение для некоторых наук» (Zeller — для действенных наук), но Никомахова «Этика» начинается словами: «Всякое искусство и всякое исследование стремится, по-видимому, к какому-нибудь благу».}


\footnotetext[46]{Дословно: признать более математической <чем должно>.}


\footnotetext[47]{Лучше, может быть, переставить: и скорее образует отличительное свойство сущности и материи, сказываясь о них, нежели материю.}


\footnotetext[48]{Указываемые «физиологами» первые различия «вещественной» материи восходят к основному устанавливаемому Платоновскою школой отличию материи — «большому и малому», потому что в них также есть характерные для большого и малого моменты избытка и недостатка.}


\footnotetext[49]{Дословно: «если указанные свойства будут движением». Под «указанными свойствами» разумеется большое и малое. Jaeger, впрочем, предлагает вместо εσται ταύτα читать έστ' έντα$$α, и тогда получится смысл: «И что касается движения, если в нашем мире есть движение, то очевидно, что идеи будут двигаться».}


\footnotetext[50]{Т.е. если согласиться, что каждому устанавливаемому роду соответствует идея. Восходя ко все более общим родам, мы тогда получим, наконец, самый общий род — сущего, который, однако, при платоновской точке зрения, будет не единым всеобъемлющим бытием, а самостоятельной, отдельно существующей идеей.}


\footnotetext[51]{Формальное различие между всеобщим и родом — то, что род всегда является чем-то общим, но не — наоборот, так что понятие общего шире понятия рода; ср. Bonitz, Comm. 299—300 (к Ζ глава 3). Как раз в понятии сущего и понятии единого Аристотель видит не род, а всеобщее.}


\footnotetext[52]{А, может быть, у элементов? См. Bon. Comm. 125 («... quod 992 b 19 scribit μή διελόντας idem significat atque hoc loco — 993 а 8 — ταύτα»). По Ross'y, I, 189 (XX) речь идет об различных значениях бытия.}


\footnotetext[53]{Дословно: «Ведь ясно, что нельзя существовать до этого <познания> (т.е. до познания элементов всего сущего), зная что бы то ни было раньше <его>».}


\footnotetext[54]{Трудно уловить оттенок различия между προειδέναι («должно знать заранее») И είναι γνώριμα (дословно: «должны быть известны»). Комментаторы, начиная с Alex., указаний не дают. Не создается ли требуемый оттенок через конкретное значение γνώριμος — «знакомый», «близкий», отсюда — понятный?}


\footnotetext[55]{Сказано по адресу Платоновского учения о припоминании в «Меноне» и «Федоне».}


\footnotetext[56]{Bonitz в Comm. (стр. 125) формулирует: «и пусть мы допустим, что у нас (благодаря прирожденному знанию) имеются такие начальные положения; все же о них поднимется спор и сомнение, которое может быть разрешено только доводами, полученными из предшествующего познания».}


\footnotetext[57]{Дословно: «и ни один из тех, которые <нам> известны». — Ross отмечает, что филологи <еще сейчас, как во времена Аристотеля, спорят, какой звук (в разные эпохи) обозначался у греков знаком ζ.}


\footnotetext[58]{Воп. и Christ принимают предложенную Schwegler'ом поправку ταύτα «из одних и тех же», вместо ταύτα — «из этих» (т. е. платоновских). Область всего бытия объясняется из пригодных для всего элементов, подобно тому, как область звуков — из элементов, пригодных для этой области.}


\end{document}

