\documentclass{article}

\usepackage[T2A]{fontenc}
\usepackage[utf8]{inputenc}
\usepackage[russian]{babel}

\linespread{1.1}
\setlength{\parskip}{1em}
\usepackage[left=1cm,right=1cm,top=1cm,bottom=2cm]{geometry}

\begin{document}

Под влиянием гераклитовского учения Платон признал невозможным познание чувственных вещей. Обратившись затем по примеру Сократа к исследованию общих понятий, он установил особые реальности --- идеи, отличные от чувственных вещей, а эти последние объявил существующими «через приобщение» к идеям. Кроме чувственных вещей и идей, он --- посредине между теми и другими --- поместил математические вещи и элементы идей признал элементами всех вещей. Указывается, что есть в учениях Платона общего с пифагорейцами и чт5 отлично от них, а также --- какими он пользовался родами причин.

\end{document}

