\documentclass{article}

\usepackage[T2A]{fontenc}
\usepackage[utf8]{inputenc}
\usepackage[russian]{babel}

\linespread{1.1}
\setlength{\parskip}{1em}
\usepackage[left=1cm,right=1cm,top=1cm,bottom=2cm]{geometry}

\begin{document}

Сходного с Анаксагором взгляда на движущую причину держались, повидимому, Гесиод и Парменид. Так как в природе порядок сохраняется не всюду, но иногда бывает и нарушен, то Эмпедокл установил двоякую причину движения: одну --- для хорошего, другую --- для дурного. Перечисленные философы устанавливали два начала: материальное и движущее, но Анаксагор движущую причину для объяснения природы использовал мало, а Эмпедокл не остается достаточно оследовательным и, помимо установления двух движущих причин, делает еще то нововведение, что у него впервые появляются четыре материальных элемента (которые при этом могут быть сведены к основным двум). Наконец, Левкипп и Демокрит признают началами вещей полное и пустое и объясняют разнообразие вещей из формы, порядка и положения элементов, а вопроса о движущей причине не ставят.

\end{document}

