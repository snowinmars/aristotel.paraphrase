\documentclass{article}

\usepackage[T2A]{fontenc}
\usepackage[utf8]{inputenc}
\usepackage[russian]{babel}

\linespread{1.1}
\setlength{\parskip}{1em}
\usepackage[left=1cm,right=1cm,top=1cm,bottom=2cm]{geometry}

\begin{document}

Разбор учений древних философов подтверждает, что, устанавливая четыре причины бытия, мы ни один род причин не пропустили. Материальное начало принимали все мыслители; некоторые добавляли также причину движения; философы, учившие идеям, з vrpoнули причину формальную; наконец, причина целевая никем не была сформулирована надлежащим образом, но в известном смысле о ней была речь у многих. Однако вне этих четырех родов причин какого-либо нового ни один философ не указал.

\end{document}

