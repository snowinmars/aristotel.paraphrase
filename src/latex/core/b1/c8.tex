\documentclass{article}

\usepackage[T2A]{fontenc}
\usepackage[utf8]{inputenc}
\usepackage[russian]{babel}

\linespread{1.1}
\setlength{\parskip}{1em}
\usepackage[left=1cm,right=1cm,top=1cm,bottom=2cm]{geometry}

\begin{document}

Имея в виду проследить, в чем древние философы правильно судили о началах, в чем --- нет, Аристотель, в первую очередь разбирает учение тех, которые установили одну материальную причину. Затем подвергается критическому разбору учение Эмпедокла. Критикуется --- а в последовательно развитом виде до некоторой степени поддерживается --- учение Анаксагора. Все эти философы рассуждали только о чувственной природе вещей. Из числа философов, которые поставили себе задачей познавать вещи, не воспринимаемые чувствами, в первую очередь, подвергаются разбору учения пифагорейцев

\end{document}

