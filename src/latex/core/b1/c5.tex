\documentclass{article}

\usepackage[T2A]{fontenc}
\usepackage[utf8]{inputenc}
\usepackage[russian]{babel}

\linespread{1.1}
\setlength{\parskip}{1em}
\usepackage[left=1cm,right=1cm,top=1cm,bottom=2cm]{geometry}

\begin{document}

Пифагорейцы полагали, что природа вещей усматривается в числах, а элементами чисел объявили чет и нечет (из коих первый знаменует в составе числа неопределенную, второй —определенную природу). Некоторые пифагорейцы признавали началами десять пар противоположностей; подобно им считал началами пары против·>положностей и Алкмеон Кротонский, не указывая однако определ иного числа этих пар. Устанавливая такие начала, пифагорейцы, повидимому, относили их к группе причин материальных. Что касается элейпев, признававших единое неизменное бытие, то их надлежит принять здесь во внимание лишь в той мере, поскольку одни из лих приписывали этому быитю логический, другие — материальный характер, а также — поскольку Пармепид пытался дать объяснение миру человеческого «мнения». Подводится итог всем дошедшим до нас взглядам древних философов на природу объясняющих мир начал и отмечается, что пифагорейцы, попидимому, впервые стали исследова ь логическую (формальную) причину, но отнеслись к этому делу очень поверхностно

\end{document}

