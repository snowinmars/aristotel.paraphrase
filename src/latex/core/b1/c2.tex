\documentclass{article}

\usepackage[T2A]{fontenc}
\usepackage[utf8]{inputenc}
\usepackage[russian]{babel}

\linespread{1.1}
\setlength{\parskip}{1em}
\usepackage[left=1cm,right=1cm,top=1cm,bottom=2cm]{geometry}

\begin{document}

а) Согласно обычным взглядам на мудрого, это — человек, который в известном смысле знает обо всем; далее — тот, кто может постигать веши трудные, и тот, кто владеет знанием более точным, а также более способен свое знание передавать; равным образом из наук большее право называться мудростью имеет та, которая представляется ценною сама по себе, и та, которая играет более руководящую роль: всем этим требованиям в наибольшей мере удовлетворяет одна паука — наука о первых началах и причинах

б) Мудрость (высшая наука) имеет не действенный, но теоретический характер; это явствует и из того, что источником, откуда она появилась, было удавление, и из того, что люди пришли к ней тогда, когда у них уже было все необходимое для удовлетворения жизненных и культурных потребностей.

в) Мудрость по справедливости можно называть божественной и потому, что она в первую очередь подобает богу, и благодаря природе познаваемого ею предмета

г) Исходя от удивления, мудрость в конечном счете приходпт к такому удивлению, которое противоположно первоначальному.

\end{document}

