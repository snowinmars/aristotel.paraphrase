\documentclass{article}

\usepackage[T2A]{fontenc}
\usepackage[utf8]{inputenc}
\usepackage[russian]{babel}

\linespread{1.1}
\setlength{\parskip}{1em}
\usepackage[left=1cm,right=1cm,top=1cm,bottom=2cm]{geometry}

\begin{document}

Итак, я описал в общих чертах, кто что говорил про начала. Все упомянутые философы называют только те начала, которые были рассмотрены в сочинении "О природе". В самом деле, все они имеют в виду одно и тоже, но используют разную терминологию. Платон говорит об избытке и недостатке; пифагорейцы - о беспредельном; Эмпедокл - об огне, земле, воздухе и воде; Анаксагор - о беспредельном множестве Гомеомерии. Все они занимаются поиском начал, но лишь некоторые - ещё и поиском причины существования вещей.

Но суть, сущность вещи никто чётко не объяснил. Ближе всего к определению сущности подошла идея эйдосов. Ведь эйдос - не материя и не двигатель изменений. Эйдосы - не причина изменений, а причина неизменности.

Но и цель изменений такие философы почему-то тоже называют причиной. Ранее я говорил, что конечная цель любого действия - благо. Те философы, кто считают ум, дружбу, единое или сущее причиной изменений в мире, одновременно с этим считают их благом - то есть целью. Но это не имеет смысла: причина не может быть ещё и целью.

Получается, что никаких других причин, кроме рассмотренных в третьей главе, никто найти не может. Предположим, что причины найдены верные или хотя бы близкие к верным. Рассмотрим проблемы, с которыми столкнулись другие философы, изучающие начала.

\end{document}

