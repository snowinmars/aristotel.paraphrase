\documentclass[oneside, 17pt, dvipsnames]{extbook}

\title{
    Парафраз "Метафизики" Аристотеля
}
\author{
    snowinmars
    \\
    Evoo
}
\date{

}

\usepackage{xltxtra}
\usepackage{polyglossia}
\setdefaultlanguage{russian}
\setotherlanguages{english, greek}
\PolyglossiaSetup{russian}{indentfirst=true}
\PolyglossiaSetup{english}{indentfirst=true}
\PolyglossiaSetup{greek}{indentfirst=true}
\PolyglossiaSetup{latex}{indentfirst=true}
\usepackage{indentfirst} % отделять первую строку раздела абзацным отступом тоже

\usepackage{epigraph}
\setlength{\epigraphwidth}{1400pt}

% colors and fonts
\usepackage{xcolor}
\definecolor{bgColor}{RGB}{17,17,17}
\definecolor{fgColor}{RGB}{204,204,214}
\definecolor{maColor}{RGB}{123,204,123}
\setromanfont[Color=fgColor]{Liberation Sans}
\setsansfont{Liberation Sans}
\defaultfontfeatures{Ligatures={TeX},Renderer=Basic}
\pagecolor{bgColor}
\everymath{\color{maColor}}
\everydisplay{\color{fgColor}}

% geometry
\linespread{1.1}
\setlength{\parskip}{1em}
\setlength{\marginparpush}{30pt}
\usepackage[paperwidth=11in,paperheight=22in,landscape,left=1cm,right=1cm,top=1cm,bottom=2cm,outer=18cm,marginparwidth=15cm,marginparsep=1cm]{geometry}
\usepackage{enumitem}
\setlist[itemize,1]{leftmargin=\dimexpr 3cm,parsep=0cm,topsep=0cm}

% start a paragraph and a margin note at the same line
\newcommand{\alignedmarginpar}[1]{
  \leavevmode
  \marginpar{\footnotesize #1}
  \ignorespaces
}

\usepackage{hyperref}

\begin{document}

\maketitle
\tableofcontents

\section{Книга 1}

\epigraph{
Описывается постепенное восхождение от чувственного восприятия к познанию принципов. Общим у всех животных является восириятие через чувства. К чувственному восприятию у некоторых животных присоединяется сохранение воспринятых образов, или память. Ряд связанных между собой воспоминаний об одной и той же вещи создает у человека опыт. Из опыта возникает искусство и наука, причем искусство превосходит опыт тем, что его внимание направлено на общее в вещах, а не на отдельные вещи, и оно познает причину, вследствие чего тот, кто владеет искусством, способен обучать. Искусствам приписывается тем больше значения, чем более они далеки от непосредственных потребностей жизни; теоретические дисциплины выше практических, а самое высшее место занимает познание принципов, или мудрость.
}{Парафраз А.В. Кубицкого}


Все люди от природы стремятся к знанию. Доказательство тому простое: нам нравится переживать новые чувства. Мы ценим картины или мелодии ради самих картин и мелодий. Ведь неважно, есть от них практическая польза или нет. Но зрительные образы мы ценим больше всего. Зрение является основой нашего способа познания. Мы предпочитаем визуализировать подготовку и к действию, и к бездействию. Именно зрением мы обнаруживаем основные различия в вещах.

Природа дала животным способность чувствовать. У одних животных чувства оставляют в памяти сильный след, а у других --- слабый. Такой след называется опытом. Животные с хорошей памятью сообразительнее и понятливее тех, у кого память плоха. Животных без слуха --- пчёл или жуков --- практически невозможно обучить. Верно и обратное: обучать проще тех животных, кто обладает и слухом, и памятью.


\alignedmarginpar{$^1$ В оригинале "искусство" может знаичить одно из трёх:
• как навык или умение, например, "искусство готовки",
• как описание вещей искусственных, неприродных,
• как синоним теоритического знания.
По всему тексту ниже термины "искусство", "навык", "умение" и "теория" в каком---то смысле взаимозаменяемы. Русский язык слишком точен, и у нас нет общего слова для такой семантической группы. В английском языке, например, используется термин "art".}
\alignedmarginpar{$^2$ \underline{Платон, "Горгий"}: "Ты опытен --- и навыки помогают тебе, ты неопытен --- и дни твои катятся по воле случая".

Альтернативный перевод --- "Опыт --- творец искусства, а неопытность --- творец удач и неудач"}
Итак, некоторые животные запоминают свой опыт. Люди же делают ещё один шаг: мы осознаём опыт через умения $^1$ и рассуждения. Опыт --- это разные воспоминаний об одном предмете. Такой опыт проявляется в умениях и науках, которые и появляются благодаря этому самому опыту. "Опыт порождает умение, а неопытность --- попытки". $^2$

Тогда появляется умение, когда опыт формирует общий взгляд на схожие предметы. Например, опыт --- "при данной болезни Каллию и Сократу помогло данное средство". Этот опыт может превратиться в умение --- "при данной болезни помогает данное средство".

В быту обладать опытом --- практическим знанием --- значит почти то же, что и обладать умением --- знанием теоретическим. Практикующие без теории преуспевают больше, чем теоретизирующие без практики. Причина этого вот в чём. Опыт --- знание единичного, а умение --- понимание общего. Всякое же действие в быту относится к единичным, частным вещам. Врач лечит не абстрактного человека, а Каллия или Сократа. Поэтому если кто---то использует теорию без практики; если кто---то познаёт общее, но не обращает внимание на единичное --- то такой человек будет часто ошибаться, когда дойдёт до реальной работы; потому что работать приходится с единичным.

\alignedmarginpar{$^3$  А тут действительно имеется в виду "искусство" как "театр/опера/...", а не "навык"?}
\alignedmarginpar{$^4$ \underline{Аристотель, "Никомахова этика", VI, 3 --- 7.}: "Наука есть приобретенная способность души к доказательствам. <...> Наука относится к сущему, искусство же – к становлению". То есть наука относится к реальному, материальному миру. Искусство --- к миру возможностей.}
\alignedmarginpar{$^5$ \underline{Лосев А.Ф., "История античной эстетики. Аристотель и поздняя классика"}: "...имеется и такая область, о которой еще нельзя сказать ни "да", ни "нет". Это и есть то, что Аристотель называет возможностью, или, возможным, "динамическим" бытием. Сказать о той вещи, которая может быть, что её вовсе нет, никак нельзя, поскольку она, хотя её пока и нет, все же может быть, то есть содержится в теоретическом разуме в какой---нибудь зачаточной, прикрытой и не вполне реальной форме. Но сказать о ней, что она действительно есть, тоже нельзя, поскольку её в настоящее время нет, хотя она может быть в другое время. Искусство относится именно к этой области полудействительности и полунеобходимости. То, что изображается в художественном произведении в буквальном смысле, вовсе не существует на деле, но то, что здесь изображено, заряжено действительностью, является тем, что задано для действительности и фактически, когда угодно и сколько угодно может быть и не только задано, но и просто дано. Это и значит, что искусство говорит не о чистом бытии, но об его становлении, об его динамике. Последнее может быть таким, что в своем развитии оно постепенно становится вероятным. Но оно может быть даже и таким, которое в своем развитии станет самой настоящей необходимостью".}
И всё же понимание предмета исходит преимущественно из теории, нежели из практики. Приобретение умения --- более мудрая вещь, чем просто наработка опыта или наблюдение. Потому что наличие умения --- это понимание причины, а наличие опыта --- не обязательно. Ведь в самом деле: кто имеет только опыт --- тот знает, \textit{что} делать, но не знает \textit{почему}. Кто владеет умением --- тот знает, почему надо делать так, а не иначе. Поэтому мы и теоретиков в каждом деле почитаем больше чистых практиков: они мудрее, так как знают причины того, что создаётся.

Некоторые ремесленники подобны неодушевлённым предметам. Они действуют неосознанно --- как огонь, который жжёт. Неодушевлённые предметы действуют в силу своей природы, ремесленники --- по привычке. Теоретики же мудры не столько благодаря умению действовать успешно, сколько благодаря пониманию причин. Умение исходит из понимания сути, а не из простого опыта. Признак знатока --- способность научить. Теоретики способны научить теории, а практики --- нет.

Далее, ни одно из чувственных восприятий мы не считаем мудростью. Хотя они и дают важнейшие знания о единичном, но они не указывают на причину. Например, чувства ничего не говорят о том, почему огонь горяч, а указывают лишь, что он горяч.

Кто первым изобрёл искусство, выработал навык и создал теорию --- тот вызвал у людей удивление. Люди удивились не только полезности изобретения, но и мудрости автора. Одни навыки служат выживанию, другие --- развлечению, а третьи служат для игр ума и заполнения досуга. Авторов последних мы склонны считать более мудрыми: ведь их знания используются не для извлечения прибыли. Такие навыки можно назвать математическими. Мы создали математические, искусственные навыки --- и только тогда мы приобрели знания не для выживания и не для удовольствия. Отмечу, что такие искусства создавались там, где у людей был досуг, и прежде всего --- жрецами в Египте.

В "Этике" я объяснил, в чём разница между искусством $^3$ и наукой. $^4$ $^5$ Цель рассуждения теперь --- показать, что существует наука, которая заниманиется причинами и началами. Имя этой науке --- мудрость. Поэтому я ранее отметил, что обучать искусству --- более \textit{мудрая} вещь, чем нарабатывать опыт. Человек искусный мудрее человека действующего, а теоретические науки мудрее прикладных наук. Таким образом, мудрость есть наука о причинах и началах.



\newpage
\section{Книга 2}

\epigraph{
а) Согласно обычным взглядам на мудрого, это --- человек, который в известном смысле знает обо всем; далее --- тот, кто может постигать веши трудные, и тот, кто владеет знанием более точным, а также более способен свое знание передавать; равным образом из наук большее право называться мудростью имеет та, которая представляется ценною сама по себе, и та, которая играет более руководящую роль: всем этим требованиям в наибольшей мере удовлетворяет одна паука --- наука о первых началах и причинах
б) Мудрость (высшая наука) имеет не действенный, но теоретический характер; это явствует и из того, что источником, откуда она появилась, было удавление, и из того, что люди пришли к ней тогда, когда у них уже было все необходимое для удовлетворения жизненных и культурных потребностей.
в) Мудрость по справедливости можно называть божественной и потому, что она в первую очередь подобает богу, и благодаря природе познаваемого ею предмета
г) Исходя от удивления, мудрость в конечном счете приходпт к такому удивлению, которое противоположно первоначальному.
}{Парафраз А.В. Кубицкого}

\alignedmarginpar{$^1$  Если ты можешь получить какой---то редкий опыт, например, залезть на гору --- это не делает тебя мудрым. Это делает тебя сильным, смелым --- но не мудрым.}
Если мы хотим овладеть наукой мудрости, то начать следует с поиска её начал. Каков стереотип мудрого человека? Тот мудр, кто:

\begin{itemize}
\item знает всё, хотя не знает каждый предмет в отдельности,
\item способен познать трудное для обычного человека. Отмечу отдельно, что восприятие свойственно всем, поэтому в нём ничего мудрого нет, $^1$
\item точен и способен научить самой мудрости --- выявлению причин.
\end{itemize}

Можно указать и свойства самой мудрости как науки. Из всех наук мудрость больше та, которая
\begin{itemize}
\item существует ради себя и ради познания, нежели та, которая существует ради извлекаемой из неё выгоды,
\item главенствует, нежели обслуживает. Ибо мудрости надлежит наставлять, а не повиноваться.
\end{itemize}

Следовательно, тот мудр, кто обладает знанием общего. Ведь в некотором смысле он знает всё, что попадает под "общее". Но "общее" познавать труднее всего: оно расположено очень далеко от непосредственного, чувственного восприятия.

Точными называются такие науки, которые явно выделяют свои начала --- аксиомы. И чем точнее и строже наука, тем меньше в ней начал. Например, арифметика строже химии. Наука тем эффективнее учит, чем больше внимания она уделяет исследованию начал. Ведь обучение --- это указание причины для каждой вещи. Схема "знание ради знания" присуща науке о том, что наиболее достойно познания. Такая наука совершенна. А наиболее достойны познания причины и начала нашего мира, то есть его аксиомы. Через них, на их основе познаётся всё остальное. Получается, самая главная наука --- та, которая познаёт цель, ради которой надлежит действовать. В каждом отдельном случае эта цель --- то или иное благо, а в общем случае --- благо вообще.

\alignedmarginpar{$^2$  Подробнее о том, почему "благо --- то есть цель существования --- это тоже причина" написало в следующей главе.}
Из всего сказанного следует, что мудрость --- это наука, исследующая \textit{самые базисные} причины и начала. Ведь и благо --- цель существования --- это тоже причина. $^2$ Но что это за наука?

\alignedmarginpar{$^3$ Есть мнение, что этот абзац не принадлежит Аристотелю. Мы не можем знать этого наверняка, но стоит обратить внимание на две вещи. Во---первые, на неравномерность распределения темы бога по "Метафизике". Во---вторых, на оторванность абзаца: тема бога возникает внезапно, цельным куском, из ниоткуда и в никуда; тогда как и до, и после у Аристотеля каждый абзац связан с соседними. Этот абзац можно выкинуть --- и логика повествования не пострадает.}
Мудрость --- чисто теоретическая наука. Это понимают даже первые философы. Философствовать людей всегда побуждает удивление. В начале мы удивляемся всему вообще. Затем это приедается, и мы продвигаемся дальше. Мы задаёмся более значительными вопросами: о смене положения Луны, Солнца, звёзд и о происхождении Вселенной. Человек удивляющийся считает себя незнающим. Получается, что люди изобрели философию, чтобы избавиться от удивления и, следовательно, от незнания. Мы стремимся к знанию ради самого этого знания; не для какой---либо пользы, а ради понимания и ради избавления от удивления. Сама история подтверждает это. Мы начали искать такое знание только когда нашли всё необходимое для жизни. Поэтому ясно, что мы ищем такое знание не ради какой---то надобности: у нас и так всё есть. Можно провести аналогию: свободным мы называем человека, который живёт ради самого себя. А наука мудрости --- единственно свободная наука, ибо она одна существует ради самой себя. $^3$

Обладание мудростью кажется выше человеческих возможностей. Ведь во многих отношениях природа людей --- рабская. Симонид говорил: "лишь бог мог бы иметь этот дар, а человеку не подобает искать несоразмерного ему знания". Наши поэты говорят, что бог будет завидовать нам, обладающим мудростью. Ведь мы будем уподобляться мудрому богу --- поэтому излишне мудрые будут несчастны. Но бог не завистлив. Мудрость следует ценить больше всех других наук. Ибо наиболее божественная наука также и наиболее ценима. А таковой может быть только мудрость, но в двояком смысле. Во---первых, божественна та из наук, которой мог бы обладать бог. Во---вторых, божественна всякая наука о божественном. Только мудрости подходят два этих пункта. Ведь бог --- это причина и начало, которые мудрость и исследует. Такой наукой мог бы обладать только бог. Таким образом, все другие науки более необходимы, нежели мудрость, но лучше --- нет ни одной.

Как я отметил ранее, всё начинается с удивления, а кончается --- утратой удивительности. При обретении мудрости отправная точка становится обыденностью. Ничему бы сильнее не удивился бы математик, как если бы диагональ квадрата оказалась бы рациональным числом.

Итак, сказано, какова природа искомой науки --- мудрости и какова цель, к которой должны привести поиски оной.






\newpage
\subsection{Отступление: диалектические пары - snowinmars}

Я опишу часть концепции Аристотеля сразу. Так следующие главы должны стать понятнее.

Аристотель говорит о причинах. Причины --- это разные грани вещи. Каждая из причин является обязательной для существования вещи.

\begin{itemize}
\item Сущность вещи --- это уникальное определение. Это информация без носителя, это идея вещи. Например, говоря о ложке, мы сразу понимаем, что это; хотя мы не знаем, какая она: деревянная или нарисованная. Говоря о меди, мы тоже имеем в виду не конкретную медь, а некий образ. Идею. Иногда говорят о сущности вещи как о "форме" для материи.
\item Материя вещи --- это конкретика о сплаве, примесях, температуре, о материале как таковом. Конкретный алюминий, который может принять форму чего угодно.
\item Откуда/почему/причина --- "...Не дерево и не медь меняют себя. Не дерево делает ложе, и не медь --- изваяние...".
\item Куда/зачем/цель --- вещь не может существовать без цели. По Аристотелю, конечная цель любой вещи или действия - благо.
\end{itemize}

Указанные первопричины делятся на пары: "суть---материя" и "причина---цель". При этом элементы пар --- противоположности --- противостоят друг другу как плюс и минус.

С другой стороны, противоположенности имеют нечто общее: ведь мы их объединили по какой---то причине. Это "общее" проявляется в обеих противоположенностях. В одной --- избытком, в другой --- недостатком. В этом заключена половина прикладной сути диалектики.

Приведу простой пример. Абстрактное понятие температуры проявляется в виде пары "огонь---холод". Огонь --- это избыток температуры, холод --- недостаток. Мне сложно найти термин для описания общего для пар "суть---материя" и "причина---цель".

Объекты, конечно же, обладают ещё кучей качеств, каждое из которых разбивается на пару противоположенностей. Однако, лишь две указанные пары общие для всех вещей вообще. Сейчас мы называем такие пары противоположностей диалектическими парами.





\newpage
\section{Книга 3}

\epigraph{
Имеется в общем четыре рода основных начал, --- это было показано в физике и подтверждается авторитетом древних философов, которые сверх этих родов не могли больше найти ни одного. Самые древние философы привимали только материальную причину вещей --- воду, воздух или другие простейшие тела; затем, побуждаемые фактическим положением дел, они стали к материи присоединять причину движения, --- за исключением тех из них, которые пришли к убеждению, что вся совокупность вещей неподвижна. После этого Анаксагор, также под влиянием действительности, впервые понял необходимость установить причину третьего рода, из которой можво было бы вывести все, что есть хорошего во всей природе, однако же он не отделил эту причину от причины движущей.
}{Парафраз А.В. Кубицкого}

\alignedmarginpar{$^2$  Сейчас мы называем это атомами. Что ты с вещью не делай (в нормальных условиях) --- а атомы ты не уничтожишь, не потеряешь и не создашь.}
\alignedmarginpar{$^3$  Здесь и несколько глав далее Аристотель активно описывает взгляды современников на физику. Я сокращу такие части: детальные взгляды греческих философов на физику уже не актуальны и, признаться, не особо понятны.}
\alignedmarginpar{$^4$  Про Фалеса передавали такую легенду (её с большой охотой повторил Аристотель). Когда Фалеса, по причине его бедности, укоряли в бесполезности философии, он, сделав по наблюдению звезд вывод о грядущем урожае маслин, ещё зимой нанял все маслодавильни в Милете и на Хиосе. Нанял он их за бесценок (потому что никто не давал больше), а когда пришла пора и спрос на них внезапно возрос, стал отдавать их внаём по своему усмотрению. Собрав таким образом много денег, он показал, что философы при желании легко могут разбогатеть, но это не то, о чём они заботятся. Аристотель подчеркивает: урожай Фалес предсказал «по наблюдению звезд», то есть благодаря знаниям. (из "Диоген Лаэртий, О жизни, учениях и изречениях знаменитых философов, I, 22 --- 44." и "Аристотель, Политика, А IV, 4, 1259 а 3.")}
Итак, надо понять начала и причины. Тогда мы поймём всё, что из них следует.

Таких начал четыре:
\begin{itemize}
\item Сущность вещи --- суть её бытия. К этой сути обращены все вопросы "почему". Когда мы найдём первое "почему" --- мы найдём причину и начало.
\item Материя вещи.
\item Причина появления.
\item Цель --- "то, ради чего". Отмечу, что цель существования --- всегда благо.
\end{itemize}

Я в достаточной мере рассмотрел эти начала в сочинении "О природе", но для цельности изложения повторю рассуждения.

Большинство первых философов считает началами лишь материальные начала. Из какой материи состоит объект, из какой возникает и в какую, погибая, превращается. Они также утверждают, что в такой цепочке трансформаций сама вещь остаётся, но изменяется в своих проявлениях. Иными словами, они утверждают, что при трансформации вещи сохраняется некое её естество $^2$. Например, Сократ в течении своей жизни изменился: из необразованного стал образованным. Мы же не говорим, что необразованный Сократ исчез, а появился другой, образованный. Сократ остаётся один и тот же: меняются его проявления. Так же, --- говорят первые философы, --- материя не возникает из ниоткуда и не исчезает в никуда.

Каждый из первых философов описывет эту идею по---своему $^3$. Фалес $^4$ --- автор идеи о том, что материя не исчезает --- утверждет, что начало --- вода. Кто---то считает, что древнейшие люди, жившие задолго до нас и первыми писавшие о богах, держались того же взгляда на природу. Что касается Гиппона, то мы пропустим его рассуждения ввиду скудности его мыслей. Анаксимен же и Диоген считают, что воздух первее воды. Гиппас из Метапонта и Гераклит из Эфеса считают, что огонь первее. Эмпедокл же считает началами четыре элемента: вода, воздух, огонь и земля. А Анаксагор из Клазомен утверждает, что начал бесконечно много.

Исходя из этого, причину существования любой вещи можно было бы считать чисто материальной. Но такое объяснение совершенно неудовлетворительно. Действительно, пусть всякое возникновение и уничтожение исходит из нескольких материальных начал. Но почему это происходит? Что причина этого? Ведь не сама же материя вызывает собственную перемену. Не дерево и не медь меняют себя. Не дерево делает ложе, и не медь --- изваяние. Что---то другое является причиной изменения. А искать эту причину --- значит искать некую иную, нематериальную причину. В моих терминах, эта причина и есть "причина появления".

Итак, часть философов считает, что начало вещи --- это какой---то один неизменный элемент, например, огонь. В конце концов, такие философы объявляют вечной и неизменной не только материю, но и всю природу целиком. Ведь им не удаётся найти эту причину изменений. Разве что Парменид несколько преуспел в этом, да и то поскольку он полагает (в некотором смысле) два вида причин.

Лучше дела идут у тех, кто признаёт началами несколько элементов. Один элемент --- например, огонь --- они рассматривают как двигатель изменений, а другой элемент --- например, воду --- как тормоз изменений.

Но на основе этих теорий мы всё равно не можем понять природу. Сама истина побуждает искать дальше. Естественно, ни огонь, ни вода, ни что---то такое не может быть причиной существования или изменения качеств объекта. Впрочем, первые философы так и не думают. Но столь же неверно считать, что всё происходит само собой. То ли Анаксагор, то ли Гермотим из Клазомен, то ли ещё кто---то говорил следующее: "Ум находится как в живых существах, так и в природе. Этот ум и есть причина миропорядка". Это намного лучше необдуманных рассуждений предшественников. Кто придерживается такого взгляда --- тот признаёт идею "совершенства" в вещах и началом, и причиной изменений. Из этой идеи выросло понятие эйдоса, о котором будет сказано ниже.





\newpage
\section{Книга 4}

\epigraph{
Сходного с Анаксагором взгляда на движущую причину держались, повидимому, Гесиод и Парменид. Так как в природе порядок сохраняется не всюду, но иногда бывает и нарушен, то Эмпедокл установил двоякую причину движения: одну --- для хорошего, другую --- для дурного. Перечисленные философы устанавливали два начала: материальное и движущее, но Анаксагор движущую причину для объяснения природы использовал мало, а Эмпедокл не остается достаточно оследовательным и, помимо установления двух движущих причин, делает еще то нововведение, что у него впервые появляются четыре материальных элемента (которые при этом могут быть сведены к основным двум). Наконец, Левкипп и Демокрит признают началами вещей полное и пустое и объясняют разнообразие вещей из формы, порядка и положения элементов, а вопроса о движущей причине не ставят.
}{Парафраз А.В. Кубицкого}

\alignedmarginpar{$^1$  Здесь и несколько глав далее Аристотель активно описывает взгляды современников на физику. Я сокращу такие части: детальные взгляды греческих философов на физику уже не актуальны и, признаться, не особо понятны.}
Можно предположить, что Гесиод первым обратился к "совершенству" как к началу и причине существования вещи $^1$. Он считет любовь или вожделение началом и пишет: "Прежде всего во Вселенной Хаос зародился, а следом --- широкогрудая Гея". Или, например, Парменид замечает, что первее всех богов Вселенная замыслила Эрота. А кто---то считает первоначалами дружбу и вражду. Потому что в природе явно есть противоположность благу. Не только красота, но и уродство. Притом в природе плохого больше, чем хорошего. И потому можно сказать, что Эмпедокл первым говорит о зле и благе как о началах. Но с оговоркой: только если причина всех благ --- само благо, а причина всех зол --- само зло.

Упомянутые философы касаются только двух начал из четырёх перечисленных в третьей главе: материи и причины появления вещи. К тому же, философы касаются их нечётко и неуверенно. Так сражаются необученные: поворачиваясь во все стороны, они наносят иногда удачные удары, но не со знанием дела. Точно так же эти философы не до конца понимают, что говорят. Ведь далее они почти не прибегают к своим началам.

Анаксагор рассматривает ум как причину изменения вещей. Но когда ему сложно объяснить, почему что---то существует, он ссылается на ум. А в остальных случаях объявляет причиной что угодно, но не ум.

\alignedmarginpar{$^2$  Эмпедокл выделяет четыре "атома" или элемента: воздух, воду, огонь и землю. А причиной изменений вещей, состоящих из этих элементов - дружбу и вражду.}
Эмпедокл считает, что причин две: дружба и вражда. $^2$ Он прибегает к своим причинам в большей мере, но всё равно недостаточно. Вдобавок у него отсутствует самосогласованность: дружба часто разделяет, а вражда --- соединяет. Когда мировое целое через вражду распадается на элементы, огонь через вражду соединяется в одно.

Но Эмпедокл впервые разделил одну движущую причину на две, притом противоположенные. Кроме того, его четыре элемента (огонь, земля, воздух и вода) впервые формируют диалектическую пару "огонь" и "не---огонь" ("земля---воздух---вода"). Такой вывод можно сделать, изучая его стихи.

А Левкипп и его последователь Демокрит признают элементами полноту и пустоту. Одно называют сущим, другое --- не---сущим. Например, полное и плотное --- сущим, а пустое и разреженное --- не---сущим. Поэтому они и говорят, что сущее существует не в большей степени, чем не---сущее. Ведь и не---пустота, и пустота состоят из материи (полной и пустой соответственно). Материальной причиной существующего они называют обе эти противоположности. Следуя учению атомистов, Левкипп и Демокрит утверждают, что отличия атомов (у них их, правда, всего два: атом пустоты и атом полноты) --- это причины всех остальных отличий. А отличия атомов бывают трёх видов:
\begin{itemize}
\item очертания --- как "А" от "Р",
\item порядок --- как "АР" от "РА",
\item положение --- как "Ь" от "Р".
\end{itemize}
А вопрос о причине изменений атомов они, подобно остальным, легкомысленно обошли.

Итак, на этом я завершу обзор исследований причин моих предшественников.





\newpage
\section{Книга 5}

\epigraph{
Пифагорейцы полагали, что природа вещей усматривается в числах, а элементами чисел объявили чет и нечет (из коих первый знаменует в составе числа неопределенную, второй --- определенную природу). Некоторые пифагорейцы признавали началами десять пар противоположностей; подобно им считал началами пары против·>положностей и Алкмеон Кротонский, не указывая однако определ иного числа этих пар. Устанавливая такие начала, пифагорейцы, повидимому, относили их к группе причин материальных. Что касается элейпев, признававших единое неизменное бытие, то их надлежит принять здесь во внимание лишь в той мере, поскольку одни из лих приписывали этому быитю логический, другие --- материальный характер, а также --- поскольку Пармепид пытался дать объяснение миру человеческого «мнения». Подводится итог всем дошедшим до нас взглядам древних философов на природу объясняющих мир начал и отмечается, что пифагорейцы, попидимому, впервые стали исследова ь логическую (формальную) причину, но отнеслись к этому делу очень поверхностно
}{Парафраз А.В. Кубицкого}

\alignedmarginpar{$^1$  Здесь и несколько глав далее Аристотель активно описывает взгляды современников на физику. Я сокращу такие части: детальные взгляды греческих философов на физику уже не актуальны и, признаться, не особо понятны.}
\alignedmarginpar{$^2$  Есть хороший пример философии Пифагора в современном кино: фильм "Пи" (1997). https://www.youtube.com/embed/TxCt093lDWw}
В это же время $^1$ --- или даже раньше --- пифагорейцы развили математику. Началами мира они стали считать её начала. А так как среди начал математики главное --- это число, то во взаимодействии чисел пифагорейцы усматривают много схожего с реальным миром. По крайней мере, больше, чем в огне, земле или воде. Например, они считают, что такое---то свойство чисел --- это справедливость; другое --- душа, или ум, или удача и так далее. Они считают, что гармония выразима в числах, и вся природа уподобляется числам. В конце концов, они предположили, что элементы чисел --- это элементы всего существующего, и что всё небо --- это гармония и число. $^2$

С помощью гармонии чисел пифагорейцы проводят аналогии со всем, с чем могут. Они самосогласовывают всю систему, а если у них в знаниях получается пробел --- они стремятся его заполнить и согласовать с остальной частью.

Их взгляды необходимо разобрать подробнее, чтобы понять, как они соотносятся с моими взглядами.

\alignedmarginpar{$^3$  Ранее по тексту термин "единое" не разъясняется. О нём будет много сказано в пятой книге. Пока можно считать "единое" самой высшей сущностью - богом?}
Пифагорейцам число "10" представляется совершенным и всеобъемлющим. Поэтому небесных тел должно быть десять. Правда, видно только девять, поэтому десятым они объявили "противоземлю". В другом сочинении об их мировоззрении рассказано подробнее. Пифагорейцы число принимают и за первоначало, и за материю, и за выражение её состояний и свойств. Элементами числа они считают чётное и нечётное. Единое $^3$ же у них и чётное, и нечётное одновременно, так как состоит из двух этих элементов.

Другие пифагорейцы утверждают, что начал десять пар:
\begin{itemize}
\item свет --- тьма
\item правое --- левое
\item примое --- кривое
\item чётное --- нечётное
\item единое --- множество
\item хорошее --- дурное
\item женское --- мужское
\item покоящееся --- движущееся
\item квадратное --- продолговатое
\item предельное --- беспредельное
\end{itemize}

Того же мнения, видимо, держится и Алкмеон из Кретона. Он утверждает, что большинство свойств образуют пары. В отличие от пифагорейцев, он имеет в виду не определённые свойства, а вообще все свойства: белое---чёрное, сладкое---горькое, хорошее---дурное, большое---малое и так далее.

От этого учения можно взять идею противоположностей. Противоположности --- это основа мира, это начала. Однако, у пифагорейцев не сказано, как свести эти противоположности к указанным в третьей главе первопричинам.

\alignedmarginpar{$^3$  Ранее по тексту термин "единое" не разъясняется. О нём будет много сказано в пятой книге. Пока можно считать "единое" самой высшей сущностью - богом?}
Мелисс считает единое $^3$ материальным и безграничным, а Парменид --- нематериальным и ограниченным. Ксенофан (говорят, Парменид был его учеником) утверждает, что единое --- это бог. Этих философов надо оставить без внимания. Ксенофана и Мелисса --- совсем, Парменида --- частично, так как говорит он с большей проницательностью. Например, он считает, что существует только сущее. Об этом яснее сказано в сочинении "О природе". Однако, сразу же за этим Парменид предлагает две причины: тёплое (огонь/сущее) и холодное (земля/не---сущее).

Подведём итоги. Все ранние философы утверждают, что
\begin{itemize}
\item одно начало --- материальное. Ведь и огонь, и вода --- это тела. Одни считают, что начал много, другие --- что мало, но все считают их телами
\item некоторые выделяют началом "причину появления". Причём, иногда эта причина одна, иногда --- пара противоположностей.
\end{itemize}

До пифагорейцев философы высказывались о началах скудно. Пифагорейцы первыми стали рассуждать про суть вещей и первыми начали искать ей определение. Но они рассматривают эту проблему слишком просто. Определения их поверхностны. Они считают сутью вещи то, что прежде всего подходит под их определение. На этом обзор ранних философов можно считать оконченным.





\newpage
\section{Книга 6}

\epigraph{
Под влиянием гераклитовского учения Платон признал невозможным познание чувственных вещей. Обратившись затем по примеру Сократа к исследованию общих понятий, он установил особые реальности --- идеи, отличные от чувственных вещей, а эти последние объявил существующими «через приобщение» к идеям. Кроме чувственных вещей и идей, он --- посредине между теми и другими --- поместил математические вещи и элементы идей признал элементами всех вещей. Указывается, что есть в учениях Платона общего с пифагорейцами и чт5 отлично от них, а также --- какими он пользовался родами причин.
}{Парафраз А.В. Кубицкого}

\alignedmarginpar{$^1$  Я буду стараться использовать термин "эйдос". Не давая прямого определения, я перечислю ряд близких к нему слов: "вид", "эталон", "образец", "чертёж". Эйдос нематериален. Эйдос отличается от идеи тем, что идея обобщает несколько вещей, а эйдос выделяет конкретику. Идея - это определение сути, эйдос - определение признаков. Скульптор может воспринять эйдос мысленно и двигать к нему свою скульптуру.}
\alignedmarginpar{$^2$  Иногда почему-то употребляют схожий термин "forms", но это не одно и то же. }
\alignedmarginpar{$^3$  Например, эйдос квадрата уникален. А вот математических квадратов существует много.}
\alignedmarginpar{$^4$  Ранее по тексту термин "единое" не разъясняется. О нём будет много сказано в пятой книге. Пока можно считать "единое" самой высшей сущностью.}
После рассмотренных философских учений появилось учение Платона. Чем---то оно схоже с пифагорейским. Платон считает, что все чувственно воспринимаемые вещи постоянно меняются, но знания об этих вещах остаются неизменными. В этом Платон схож с Кратилом и вообще с гераклитовскими воззрениями. Учитель Платона Сократ первым обратил внимание на важность определений. И Платон, следуя своему учителю, считает, что определения относятся к не---чувственно воспринимаемому миру. Ибо, --- говорит Платон, --- нельзя дать определения чувственно воспринимаемому, поскольку оно постоянно изменяется. Такие принципиально неизменные определения Платон называет эйдосами. $^1$ $^2$. Эйдос виртуален, а реальная вещь --- это воплощение виртуального эйдоса. Пифагорейцы утверждают то же самое, но другими словами. У них --- "вещи существуют через подражание числам", у Платона --- "вещи существуют через причастность эйдосам". Но они не определяют, что такое подражание/причастность.

Платон утверждает, что существуют ещё и математические предметы. Они отличаются от вещей тем, что они вечны и неизменны, а от эйдосов --- тем, что существует много одинаковых математических предметов, а каждый эйдос уникален. $^3$

Эйдосы, конечно же, состоят из элементов. Платон полагает, что эти элементы (чем бы они ни были) --- это элементы всего остального. Материальное начало он обозначает как пару "избыток---недостаток", а причиной изменений называет единое. $^4$ Платон считает, что эйдосы получаются из избытка и недостатка через причастность единому. $^5$ $^6$ То есть: пусть существует некое Единое --- какое---то интересующее нас качество. Это качество проявит себя в виде пары противоположностей: одна противоположность будет избытком качества, другая --- скудностью качества. Получается, что противоположности --- это...противоположенные друг другу вещи, но они являются частным проявлением одного и того же качества, то есть причастны ему.

В отличии от Платона, пифагорейцы не считают, что математические объекты существуют отдельно от вещей и эйдосов. $^7$

Легко показать, что взгляд пифагорейцев не основателен. Они полагают, что из одной единицы материи создаётся много вещей, каждой из которых соответствует свой эйдос.
\alignedmarginpar{$^5$  В "Метафизике" я не увидел явного определения причастности, но по тексту встречаются определяющие примеры. Очень грубо и кратко: причастный - это что-то вроде подкласса. Хорошее слово - "сиюминутный".
- - "Человек" причастен "живому существу".

- - "Человек" - подкласс "живых существ".

- - "Человек" - сиюминутное "живое существо".

-

- - "Сократ" причастен "человеку" и "живому существу".

- - "Сократ" - полное пересечение классов "людей" и "живых существ".

- - "Сократ" - сейчас "живой" и "человек", а потом - нет.

-

- - "Бледный человек" причастен "бледности" и "человеку" одновременно.

- - "Бледный человек" - частичное пересечение классов "бледных" и "людей".

- - "Бледный человек" - сиюминутно "бледный" сиюминутный "человек".}
\alignedmarginpar{$^6$  Сейчас будет моя отсебятина, но, вроде, адекватная.}
\alignedmarginpar{$^7$  Я не могу это пересказать так, чтобы я сам понял.}
Однако, очевидно, что из одной единицы материи получается только один стол. И что множеству таких столов соответствует один и тот же эйдос. Это как с мужским и женским: женское оплодотворяется одним мужским, а мужское оплодотворяет многие женские.

Из сказанного ясно, что Платон рассматривает только две причины. В терминах третьей главы это "сущность вещи" (эйдос) и "материя вещи" (пара "избыток---недостаток"). Для эйдосов причиной появления он указывает единое. И для эйдосов, и для вещей Платон указывает материальным элементом пару "избыток---недостаток". Кроме того, он объявляет эти элементы причиной блага и зла соответственно, как в своё время это же сделали Эмпедокл и Анаксагор.






\newpage
\section{Книга 7}

\epigraph{
Разбор учений древних философов подтверждает, что, устанавливая четыре причины бытия, мы ни один род причин не пропустили. Материальное начало принимали все мыслители; некоторые добавляли также причину движения; философы, учившие идеям, з vrpoнули причину формальную; наконец, причина целевая никем не была сформулирована надлежащим образом, но в известном смысле о ней была речь у многих. Однако вне этих четырех родов причин какого-либо нового ни один философ не указал.
}{Парафраз А.В. Кубицкого}

Итак, я описал в общих чертах, кто что говорил про начала. Все упомянутые философы называют только те начала, которые были рассмотрены в сочинении "О природе". В самом деле, все они имеют в виду одно и тоже, но используют разную терминологию. Платон говорит об избытке и недостатке; пифагорейцы --- о беспредельном; Эмпедокл --- об огне, земле, воздухе и воде; Анаксагор --- о беспредельном множестве Гомеомерии. Все они занимаются поиском начал, но лишь некоторые --- ещё и поиском причины существования вещей.

Но суть, сущность вещи никто чётко не объяснил. Ближе всего к определению сущности подошла идея эйдосов. Ведь эйдос --- не материя и не двигатель изменений. Эйдосы --- не причина изменений, а причина неизменности.

Но и цель изменений такие философы почему---то тоже называют причиной. Ранее я говорил, что конечная цель любого действия --- благо. Те философы, кто считают ум, дружбу, единое или сущее причиной изменений в мире, одновременно с этим считают их благом --- то есть целью. Но это не имеет смысла: причина не может быть ещё и целью.

Получается, что никаких других причин, кроме рассмотренных в третьей главе, никто найти не может. Предположим, что причины найдены верные или хотя бы близкие к верным. Рассмотрим проблемы, с которыми столкнулись другие философы, изучающие начала.






\newpage
\section{Книга 8}

\epigraph{
Имея в виду проследить, в чем древние философы правильно судили о началах, в чем --- нет, Аристотель, в первую очередь разбирает учение тех, которые установили одну материальную причину. Затем подвергается критическому разбору учение Эмпедокла. Критикуется --- а в последовательно развитом виде до некоторой степени поддерживается --- учение Анаксагора. Все эти философы рассуждали только о чувственной природе вещей. Из числа философов, которые поставили себе задачей познавать вещи, не воспринимаемые чувствами, в первую очередь, подвергаются разбору учения пифагорейцев
}{Парафраз А.В. Кубицкого}

Кто считает, что всё материально --- тот явно ошибается. В самом деле, такой философ указывает элементы только для материальных объектов, но не указывает элементы нематериальных. Хотя нематериальные объекты, безусловно, существуют и, следовательно, состоят из неких элементов. Точно так же, пытаясь указать причины возникновения и уничтожения вещей, такие философы упускают из вида нематериальную причину изменений.

\alignedmarginpar{$^1$  Для нас сейчас - это кварки.}

Более того, такие философы необдуманно называют началом любые простые тела, не указывая, как возникают эти тела друг из друга. Например, как вода возникает из огня? Ведь вещи возникают либо через соединение, либо через разъединение. Это очень важно потому, что помогает понять причинно---следственную связь. Можно утверждать, что самый базисный элемент --- это тот, из которого возникают через соединение все вещи. $^1$ Таким могло бы быть множество мельчайших частиц. Ближе всего к этому взгляду расположена огненная первопричина. Каждый из философов согласен с этим. По крайней мере, никто не пытался указывать первоосновой землю. Ведь она явно состоит из крупных частиц. А из трёх других элементов, как раз и состоящих из мелких частиц, каждый нашёл себе сторонника.

\alignedmarginpar{$^2$  Напомню, что говорится об Эмпедокле в третьей главе. Он выделяет четыре элемента: воду, воздух, огонь и землю; а причиной изменений вещи считает диалектическую пару "дружба-вражда".}
То же можно сказать и о тех, кто признаёт несколько начал. Например, Эмпедокл утверждает, что начал четыре. $^2$ Поэтому у него должны получаться отчасти --- те же самые, отчасти --- свои собственные затруднения. В самом деле, пусть элементы возникают друг из друга. Получается, огонь и земля не всегда остаются собой: они превращаются друг в друга --- об этом сказано подробнее в сочинении "О природе". А про причину и цель изменений в мире у Эмпедокла не сказано совсем ничего правильного или обоснованного.

И вообще, те, кто разделяет идею Эмпедокла, вынуждены отвергать превращение как таковое. Ибо не может у них получиться ни холодное из тёплого, ни тёплое из холодного. В самом деле, пусть Эмпедокл прав. Тогда должно существовать какое---то одно естество, которое проявляется в теле теплом или холодом. Раз оно проявляется теплом (огонь) или холодом (вода), то оно проще и огня, и воды. А это невозможно, потому что огонь и воду Эмпедокл называет максимально простыми элементами.

\alignedmarginpar{$^3$  Напомню, что говорится об Анаксагоре в третьей главе. Он считает, что элементов бесконечно много, а причиной изменений считает ум.}
Что касается Анаксагора, то он с необходимостью получает те же проблемы трансформации. Он считает, что существует две причины изменений $^3$: ум и всё остальное. При этом "всё остальное" у него не имеет качеств. Оно не имеет ни цвета, ни вкуса. Оно не может быть ни качеством, ни количеством. Ведь если бы это было не так --- это имеющееся качество можно было бы выделить в отдельную сущность, а Анаксагор утверждает, что так сделать нельзя. Ум, по Анаксагору, несмешан и чист. Исходя из этого, Анаксагор должен признавать два начала: несмешанное (ум) и смешанное (всё остальное). Хоть он и выражает свои мысли неясно, однако он хочет сказать что---то близкое к современной мне философии.

Что касается пифагорейцев, то они рассуждают о более необычных началах, нежели другие школы философии. Ведь они заимствуют их не из чувственно воспринимаемого мира. И всё же они постоянно рассуждают о природе и исследуют её. Они прибегают к своим началам для создания теории возникновения неба и астрономической теории. Таким образом, они как бы соглашаются с другими философами в том, что сущее --- это лишь то, что воспринимается чувствами. Ведь они исследуют то, что воспринимается чувствами --- а это и есть сущее. Однако же, повторюсь, их причины и начала более пригодны к исследованию абстрактного, нежели к исследованию природы. С другой стороны, они ничего не говорят о том, почему возникают изменения. Равно как и том, как возникновение, превращение и уничтожение небесных тел возможно без теории изменений.

Далее, предположим, что из начал пифагорейцев образуется мир. Как всё же получается так, что одни тела лёгкие, а другие тяжёлые? Они же рассуждают об абстракциях. О реальных физических объектах они не говорят. Ведь пифагорейцы не в состоянии сказать ничего, присущего только физическим объектам.

Далее, как это понять, что "свойства числа и само число суть причина того, что существует и совершается на небе изначала и в настоящее время"? Что означает, что "нет никакого другого числа, кроме числа, из которого составилось мироздание"? Напомню, что пифагорейцы соотносят качества мира с числами, а числам приписывают связи с материальными объектами. Например, сверху слева расположена удача, а справа снизу --- несправедливость. Так как такие координатные системы неудобны, то пифагорейцы привязывают точки отсчёта к материальным объектам: к планетам или здёздам. А что делать, если такой объект меняется? Число остаётся там же, "на небе" или двигается вслед за объектом?

Современные мне философы склонны рассуждать только о возникновении, уничтожении или изменеии вещи. Именно для этого они и ищут начала и причины. В следующей главе остановимся на платонистах --- философах, что рассматривают всё сущее в совокупности. Напомню, что они материальные вещи признают чувственно воспринимаемыми, а эйдосы --- чувственно невоспринимаемыми.







\newpage
\section{Книга 9}

\epigraph{
Имея в виду подвергнуть крптике учение Платона, Аристотель отмечает, что гипотеза, устанавливающая идеи, опрометчиво удваивает число вещей, которые надо объяснить; затем он в первую очередь разбирает аргументы, которые Платон выдвигал в пользу существования идеи·, и констатирует в них различные логические дефекты: в одних вывод не следует с (силлогистическою) необходимостью, другие идут дальше, чем хотел сам философ. Далее он разбирает взаимоотношение между идеями и чувственнымн вещами и вопрос о том, какое воздействие производят идеи на чувственные вещи. После этого он переходит к природе чисел (к которым Платон пытался свести идеи) и излагает те трудности, которые в этом случае получаются как при попытках формулировать существо идей --- чисел, так и при объяснении из них вещей. В результате Аристотель делает вывод, что в платоновской школе весь вопрос был поставлен самым превратным образом: были отвергнуты вещи и причины, непосредственно данные, и философы установили сокровенные и совершенно бесплодные начала; их доказательства не доказывают, чего они хотят, и природу геометрических объектов нельзя совместить с характером всего учения в целом. Наконец, в виду того, что они установили одни и те же начала для всех получается, что их нельзя ни познавать, ни считать прирожденными человеку от природы
}{Парафраз А.В. Кубицкого}

Оставим пифагорейцев. Рассмотрим платонистов --- философов, что признают началами эйдосы. Они странно рассуждают: для поиска причин существования окружающих нас объектов они придумывают эйдосы, которых не меньше, чем самих объектов. И, как будет показано далее, каждый эйдос --- это тоже объект. Такой подход всего лишь отодвигает проблему на один шаг: вместо изучения вещей они предлагают изучать эйдосы.

\alignedmarginpar{$^1$  "Объектом знания должно быть нечто устойчивое и общее. А так как чувственно воспринимаемые предметы преходящи и единичны, то предметом знания могут быть только эйдосы".}
\alignedmarginpar{$^2$  "Раз существует эйдос для предмета, то должен существовать эйдос и для отрицания предмета". Например, если есть эйдос для человека, то должен быть эйдос и для нечеловека.}
\alignedmarginpar{$^3$  "Эйдосы должны существовать для предметов, которые могут всецело исчезнуть, но сохраниться в человеческой мысли". Но то же самое можно отнести и к единичным предметам и утверждать, что по той же причине для каждого из них должен существовать особый эйдос.}
\alignedmarginpar{$^4$  "Есть эйдос реального человека. Есть эйдос абстрактного человека. Так как между реальным и абстрактным человеком есть нечто общее, то у этого общего тоже есть эйдос". Тут начинается цепная реакция: сравнение порождённого эйдоса и изначальным порождает новый эйдос - и так до бесконечности.}
Далее, ни одно доказательство существования эйдосов не убедительно. Доказательства платонистов приводят к тому, что эйдосы появляются для того, для чего их быть не может. Иногда в их доказательствах вообще нет логики.
\begin{itemize}
\item Первое доказательство существования эйдосов --- доказательство от знания. $^1$ Согласно ему, эйдосы должны иметься для всего, что мы знаем.
\item Согласно доказательству "единого во многом" $^2$ эйдосы должны существовать и для отрицаний.
\item А на основании доказательства "об исчезновении" $^3$ эйдосы должны существовать и для того, что может исчезнуть из материального мира.
\item Иногда приводят довод о третьем человеке $^4$.
\end{itemize}
Эти доказательства противоречат друг другу.

\alignedmarginpar{$^5$  Я не понимаю, как оно так противоречит, ну да ладно.}
И, вообще говоря, доводы в пользу существования эйдосов сводят на нет то, что для Платона важнее самих эйдосов. Из этих доводов следует $^5$, что диалектическая пара "избыток---недостаток" не явзяется началом. То есть следствие существование эйдосов противоречит изначальному утверждению о том, что эйдосы формируются через пару "изыбток---недостаток".

Далее, предположим, что эйдосы и правда существуют. Тогда должны существовать эйдосы не только привычных нам вещей, но и многого другого. Наше мышление работает одинаково с любыми вещами: как с привычными, так и с непривычными. Да и науки посвящаются не только существующим вещам. Однако, учение об эйдосах утверждает, что эйдосы доступны нашему пониманию. Значит, эйдосы существуют только для тех вещей, о которых мы можем подумать.

С другой стороны, это учение говорит, что вещь не может существовать без эйдоса. Ведь, познавая вещь, мы неизбежно познаём и её эйдос. Например, одно из свойств противоположности --- её вечность. Приобщаясь, познавая противоположность, мы приобщаемся и к вечности, поскольку вечность --- свойство противоположности. Так же и с самой вещью: как только мы начинаем думать о вещи, мы сразу же начинаем думать и о её эйдосе. Получается, вещи без эйдоса просто нет.

Раз вещь --- это объект и вещи причастны эйдосам, то класс вещей --- это подкласс эйдосов. То есть эйдосы --- это тоже объекты. И эйдосы, и вещи --- это объекты. Иначе теория множеств не имеет смысла. Ведь элементы класса и элементы подкласса должны быть одного типа. Если же между вещами и эйдосами нет ничего общего, то это странно. Это было бы похоже на то, как если бы кто---то назвал бы и Каллия, и дерево человеком, не увидев между ними ничего общего.

Однако, сложенее всего ответить на следующий вопрос. Какое значение имеют эйдосы для вечных вещей реального мира (например, света) или для преходящих вещей (например, цветка)? Дело в том, что эйдос такой вещи --- это не причина изменения этой вещи. С другой стороны, эйдос ничего не говорит о самой вещи (ведь он находится вне её). А с третьей стороны, эйдос ничего не говорит о других вещах: иначе он был бы в этих других вещах. Можно подумать, что эйдосы --- это причины изменений вещей в том же смысле, в каком примешивание белого цвета к объекту делает его белым. Этот довод выдвинул сначала Анаксагор, а потом --- Евдокс. На самом деле, он очень шаток, и его легко развенчать.

Вместе с тем, вещи вообще не могут происходить из эйдосов ни в каком смысле. Говорить, что эйдосы --- это образцы, и что всё остальное им причастно --- значит пустословить и говорить иносказаниями. В самом деле, что это означает? Ведь можно и быть, и становиться схожим с чем угодно, не подражая образцу. Не важно, существует сейчас Сократ или нет. Всегда может появиться человек, сколь угодно похожий на Сократа. Это означает то же самое, как если бы существовал вечный Сократ.

Есть и второй вариант. Должно быть множество эйдосов для одного и того же объекта. Это означает, что для одного и того же человека будут существовать эйдос человека, эйдос живого существа, эйдос двуногого и так далее.

Далее, так как эйдосы --- это объекты, а у объектов есть эйдос, то у каждого эйдоса тоже должен быть эйдос.

Далее, очевидно, что эйдос и объект не могут существовать отдельно друг от друга. Между тем, в "Федоне" утверждается именно это: эйдосы --- это причины бытия и возникновения вещей. Однако, объект, даже имея эйдос, не возник бы, если бы не было того, что заставило его возникнуть. С другой стороны, утверждается, что искуственно созданные человеком объекты не имеют эйдосов. Например, современный дом или кольцо. Но они точно есть --- сам мир противоречит этому утверждению.

Далее, если эйдосы --- это числа, то как они могут быть причиной изменений? Ну, предположим, что вещи --- это числа: человек --- вот это число, Сократ --- вот это, а это число --- Каллий. Значит, эйдос --- число и объект --- тоже число. Как тогда так получается, что одни числа --- это причины для других чисел? И то, и то --- числа, а числа --- это нечто вечное. Это просто не имеет смысла.

Тогда предположим, что числа --- это причины изменений потому, что окружающие нас вещи описываются через числовые отношения. Каждой вещи сопоставим дробь. Тогда выходит, что одной вещи сопоставляются две других: числитель и знаменатель. Получается бесконечная лестница. Если Каллий --- это некое соотношение огня и воды, то и его эйдос --- это соотношение чего---то ещё и чего---то ещё, и так далее: вся бесконечная цепочка будет состоять из соотношений вещей, а не из чисел, как утверждалось изначально.

Далее, опыт подсказывает, что из многих чисел можно собрать одно число. Но как может из многих эйдосов получиться один эйдос? Можно сказать, что сумма получается не из самих чисел, а из их составляющих --- единиц, но это ничего не объясняет. Такое утверждение лишь отодвигает проблему на один шаг. Ведь единица --- это тоже число, что с ними делать? Все это не основательно и не согласуется с нашим мышлением. Кроме того, такой подход требует признать какой---то другой вид чисел, отличный от того, с каким имеет дело арифметика.

Затем, философия эйдосов считает, что всё образовывается из диалектической пары. Из этого следует, что каждая из противоположенностей этой пары, образовывается из некоторой предшествующей диалектической пары. А это невозможно, потому что порождает бесконечность.

Далее, если просто число по определению едино, то почему составное число тоже едино?

Далее, можно сказать, что все единицы в числе различны. Но тогда уж надо говорить прямо, как те философы, кто утверждает, что элементов два, четыре или больше. Ведь каждый называет своим элементом не тело вообще, а конкретно огонь или землю. Вне зависимости от того, есть ли что---то общее для огня и земли или нет. Платон говорит о едином так, как будто оно состоит из чисел как из однородных частиц. В таком случае, числа не могут быть сущностями. Сущности должны чем---то отличаться, а однородные элементы неотличимы по определению. Это противоречит его утверждению о том, что числа --- это эйдосы.

Как, согласно Платону, начала формируют сущности? Через синтез "избытка" и "недостатка". Линия --- из длинного и короткого, плоскость --- из широкого и узкого, тело --- из высокого и низкого. Однако, непонятно, как в таком случае плоскость будет содержать линию? Ведь "широкое и узкое" относится к другому роду, нежели "высокое и низкое". Получается, что в пространства больших размерностей не могут содержаться пространства меньших размерностей --- то есть плоскость не может содержать линию, а это неверно.

Далее, что такое точка для фигуры? Правда, Платон решительно возражал против признания точки самостоятельный родом навроде линии. Он считал точку геометрическим вымыслом. А началом линии он называл "неделимые линии". Очевидно, что раз существуют неделимые линии, то существуют и точки: разницы нет.

Платон говорит, что эйдос определяет суть вещи. На самом деле это замкнутый цикл. Мы хотим найти сущность вещи --- и для этого вводим другие вещи. Но мы ничего не говорим о механизме такого определения. Платон называет такой механизм причастностью, но выше было показано, что причастность не означает ничего. Так как мудрость работает только с реальными объектами, то этот вопрос --- вопрос о природе математических объектов --- останется без внимания.

Эйдосы не описывают причину поиска знаний, ради которой творит всякий ум. Математика заменила нынешним мудрецам философию, хотя цели у этих наук совершенно разные.

Материя у Платона --- диалектическая пара "избыток---недостаток" --- это не совсем материя. Эта пара описывает очень уж абстрактные вещи, и говорит больше о видовых различиях, нежели о материи как таковой. Если бы избыток и недостаток были бы подвержены изменениям, то и эйдосы должны были бы меняться со временем. Ведь эйдосы тоже состоят из этой диалектической пары. А если избыток и недостаток не меняются сами и не создают причин для изменения вещей, то откуда берётся изменение в природе?

\alignedmarginpar{$^6$  Я пропустил абзац: совсем непонятно.}
Также и то, что кажется легким делом, --- доказать, что все едино, этим способом не удается, ибо через отвлечение (ekthesis) получается не то, что все едино, а то, что есть некоторое само---по---себе---единое, если даже принять все [предпосылки]. Да и этого самого---по---себе---единого не получится, если не согласиться, что общее есть род; а это в некоторых случаях невозможно.

\alignedmarginpar{$^7$  Фигня какая-то. Наверняка Аристотель под "математическими объектами" имел в виду что-то своё, древнее. Со стороны современной математики, и линия, и число - это математический объект.}
У Платона также никак не объясняется, почему и зачем существует что---то кроме чисел: линии, плоскости и тела. С одной стороны, такие объекты не могут быть эйдосами потому, что не являются числами. С другой --- не могут быть и математическими объектами. $^7$ А с третьей --- не могут быть вещами реального мира. Получается, линия --- это какой---то четвёртый род сущностей?

К слову, бесполезно искать элементы для "всего вообще". Сущее очень многообразно и сложно. В самом деле, из каких общих элементов могут состоять "действие" и "прямота"? Элементы можно указать для физических объектов, но не для математических. Поэтому неправильно искать элементы \textit{всего} сущего: следует ограничиться лишь материальной его частью.

Да и как можно было бы познать элементы всего? Ясно одно: что пока мы не познаем элементы всего, мы не сможем познать ничего вообще. Например, мы так же учимся геометрии. Мы можем заранее знать пару фактов. Но мы не можем заранее знать ни то, чему посвящена геометрия; ни то, что мы будем в ней учить. Поэтому человек не может иметь знаний о мире до того, как он начнёт изучение науки обо всём существующем --- мудрости. Всякое изучение начинается с указания аксиом и определений, и только потом --- доказательств. Потому что доказательства строятся на аксиомах и определениях. Получается, если мы уже знаем что---то о мире, то мы уже знаем элементы всего? Это так же абсурдно, как если бы мы знали геометрию до того, как приступили к её изучению. С другой стороны, если бы оказалось, что знание элементов всего является для человека врождённым знанием, то это было бы удивительно. Как же оставалось незамеченным обладание наилучшим из знаний?

Далее, как вообще узнать, из каких элементов состоит сущее? Это сложный вопрос. Здесь можно спорить как о слогах: одни говорят, что "ща" состоит из звуков "щ" и "а"; а другие --- что "ща" --- это вообще отдельный звук.

Если бы всё заключалось только в познании "элементов всего", то слепой, познав бы эти элементы, понял бы цвет. Но как можно познать цвет, не имея зрения?







\newpage
\section{Книга 10}

\epigraph{
Никто из прежних философов не установил какого-либо начала, кроме указанных выше четырех родов причин, но и эти причины они скорее смутно предугадали, чем познали.
}{Парафраз А.В. Кубицкого}

Из сказанного ранее ясно, что философы ищут те же причины, что обозначены мной в сочинении "О природе" и упомянуты в третьей главе "Метафизики". Помимо этих причин я не могу указать ни одной.

Но философы $^1$ выражались нечётко. То, что говорит прежняя философия, больше похоже на лепет. Философия была молода.

\end{document}
