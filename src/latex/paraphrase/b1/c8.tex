\documentclass{article}

\usepackage[T2A]{fontenc}
\usepackage[utf8]{inputenc}
\usepackage[russian]{babel}

\linespread{1.1}
\setlength{\parskip}{1em}
\usepackage[left=1cm,right=1cm,top=1cm,bottom=2cm]{geometry}

\begin{document}

Кто считает, что всё материально - тот явно ошибается. В самом деле, такой философ указывает элементы только для материальных объектов, но не указывает элементы нематериальных. Хотя нематериальные объекты, безусловно, существуют и, следовательно, состоят из неких элементов. Точно так же, пытаясь указать причины возникновения и уничтожения вещей, такие философы упускают из вида нематериальную причину изменений.

Более того, такие философы необдуманно называют началом любые простые тела, не указывая, как возникают эти тела друг из друга. Например, как вода возникает из огня? Ведь вещи возникают либо через соединение, либо через разъединение. Это очень важно потому, что помогает понять причинно-следственную связь. Можно утверждать, что самый базисный элемент - это тот, из которого возникают через соединение все вещи.
\footnotemark[1]
Таким могло бы быть множество мельчайших частиц. Ближе всего к этому взгляду расположена огненная первопричина. Каждый из философов согласен с этим. По крайней мере, никто не пытался указывать первоосновой землю. Ведь она явно состоит из крупных частиц. А из трёх других элементов, как раз и состоящих из мелких частиц, каждый нашёл себе сторонника.

То же можно сказать и о тех, кто признаёт несколько начал. Например, Эмпедокл утверждает, что начал четыре.
\footnotemark[2]
Поэтому у него должны получаться отчасти - те же самые, отчасти - свои собственные затруднения. В самом деле, пусть элементы возникают друг из друга. Получается, огонь и земля не всегда остаются собой: они превращаются друг в друга - об этом сказано подробнее в сочинении "О природе". А про причину и цель изменений в мире у Эмпедокла не сказано совсем ничего правильного или обоснованного.

И вообще, те, кто разделяет идею Эмпедокла, вынуждены отвергать превращение как таковое. Ибо не может у них получиться ни холодное из тёплого, ни тёплое из холодного. В самом деле, пусть Эмпедокл прав. Тогда должно существовать какое-то одно естество, которое проявляется в теле теплом или холодом. Раз оно проявляется теплом (огонь) или холодом (вода), то оно проще и огня, и воды. А это невозможно, потому что огонь и воду Эмпедокл называет максимально простыми элементами.

Что касается Анаксагора, то он с необходимостью получает те же проблемы трансформации. Он считает, что существует две причины изменений
\footnotemark[3]
: ум и всё остальное. При этом "всё остальное" у него не имеет качеств. Оно не имеет ни цвета, ни вкуса. Оно не может быть ни качеством, ни количеством. Ведь если бы это было не так - это имеющееся качество можно было бы выделить в отдельную сущность, а Анаксагор утверждает, что так сделать нельзя. Ум, по Анаксагору, несмешан и чист. Исходя из этого, Анаксагор должен признавать два начала: несмешанное (ум) и смешанное (всё остальное). Хоть он и выражает свои мысли неясно, однако он хочет сказать что-то близкое к современной мне философии.

Что касается пифагорейцев, то они рассуждают о более необычных началах, нежели другие школы философии. Ведь они заимствуют их не из чувственно воспринимаемого мира. И всё же они постоянно рассуждают о природе и исследуют её. Они прибегают к своим началам для создания теории возникновения неба и астрономической теории. Таким образом, они как бы соглашаются с другими философами в том, что сущее - это лишь то, что воспринимается чувствами. Ведь они исследуют то, что воспринимается чувствами - а это и есть сущее. Однако же, повторюсь, их причины и начала более пригодны к исследованию абстрактного, нежели к исследованию природы. С другой стороны, они ничего не говорят о том, почему возникают изменения. Равно как и том, как возникновение, превращение и уничтожение небесных тел возможно без теории изменений.

Далее, предположим, что из начал пифагорейцев образуется мир. Как всё же получается так, что одни тела лёгкие, а другие тяжёлые? Они же рассуждают об абстракциях. О реальных физических объектах они не говорят. Ведь пифагорейцы не в состоянии сказать ничего, присущего только физическим объектам.

Далее, как это понять, что "свойства числа и само число суть причина того, что существует и совершается на небе изначала и в настоящее время"? Что означает, что "нет никакого другого числа, кроме числа, из которого составилось мироздание"? <span class="help">Напомню, что пифагорейцы соотносят качества мира с числами, а числам приписывают связи с материальными объектами. Например, сверху слева расположена удача, а справа снизу - несправедливость. Так как такие координатные системы неудобны, то пифагорейцы привязывают точки отсчёта к материальным объектам: к планетам или здёздам. А что делать, если такой объект меняется? Число остаётся там же, "на небе" или двигается вслед за объектом?</span>

Современные мне философы склонны рассуждать только о возникновении, уничтожении или изменеии вещи. Именно для этого они и ищут начала и причины. В следующей главе остановимся на платонистах - философах, что рассматривают всё сущее в совокупности. Напомню, что они материальные вещи признают чувственно воспринимаемыми, а эйдосы - чувственно невоспринимаемыми.

\end{document}

